%%% LaTeX Template: Article/Thesis/etc. with colored headings and special fonts
%%%
%%% Source: http://www.howtotex.com/

\documentclass[12pt]{article}


\usepackage{apuntes-estilo}
\usepackage{fancyhdr,lastpage}



\def\maketitle{

% Titulo 
 \makeatletter
 {\color{bl} \centering \huge \sc \textbf{
Trabajo práctico N 4 \\
\large \vspace*{-8pt} \color{black} Aplicaciones Multimedia 
 \vspace*{8pt} }\par}
 \makeatother


% Autor
 \makeatletter
 {\centering \small 
	Aplicaciones Libres\\
 	Departamento de Ingeniería de Computadoras \\
 	Facultad de Informática - Universidad Nacional del Comahue \\
 	\vspace{20pt} }
 \makeatother

}

% Custom headers and footers
\fancyhf{} % clear all header and footer fields
\fancypagestyle{plain}{\fancyhf{}}
  	\pagestyle{fancy}
 	\lhead{\footnotesize TP N 4 - Aplicaciones Libres}
 	\rhead{\footnotesize \thepage\ }	% "Page 1 of 2"

\def\ti#1#2{\texttt{#1} & #2 \\ }



\begin{document}

\thispagestyle{empty}
\maketitle
\setlength{\parindent}{0pt}


\section*{Introducción}
El mundo de las aplicaciones multimedia es inmenso. Como administradores de 
sistemas encontraremos cientos de requisitos diferentes sobre imagen, audio, 
video y composición multimedial. 

Este trabajo tendrá varios objetivos con respecto a multimedia:
\begin{itemize} 
\item Conocer terminología básica respecto a cada componente multimedial. 
\item Conocer la problemática básica de las aplicaciones. Cómo se agrupan, cuáles
son las funcionalidades principales. 
\item Visualizar los recursos necesarios a nivel de hardware y software.
\item Vincular aplicaciones con los formatos de archivos manipulados. 
\item Vincular aplicaciones de software libre con aplicaciones utilizadas 
como estándar de facto, posiblemente privativas. 
\end{itemize}

\section*{Descripción}
Es habitual que comunidades de personas con fines comunes aúnen esfuerzo para 
resolver ciertos tipo de problemas, el mundo del software, y en particular el 
mundo de los sistemas operativos GNU/Linux no es ajeno a este fenómeno. En 
este caso, y con los objetivos mencionados anteriormente, analizaremos el 
trabajo de comunidades de usuarios y desarrolladores de aplicaciones 
multimediales, que han dado como resultado nuevas distribuciones GNU/Linux.  

Nuestro objetivo no será promover el uso de las distribuciones en sí, sino el 
análisis de las mismas como medio de obtener conocimiento acerca de 
aplicaciones multimedia instaladas en ellas y la organización de las 
mismas (forma en que están agrupadas y presentadas a los usuarios). El 
administrador podrá considerar en el futuro este tipo de soluciones cuando 
se adapte a los requisitos de los usuarios.  

Las distribuciones a considerar serán: 
\begin{itemize}
\item Ubuntu Studio \url{http://ubuntustudio.org/}
\item DreamStudio (formerly Dream Studio) \url{http://www.celeum.com/about-dreamstudio/}
\item ArtistX \url{http://artistx.org/blog/}
\item Musix GNU+Linux \url{https://musixdistro.wordpress.com/}
\item AV Linux \url{http://www.bandshed.net/AVLinux.html}
\end{itemize}


\subsection*{Características básicas de la distribución:}

La siguiente información deberá ser recolectada para la distribución asignada: 

\begin{enumerate}
\item País de origen. 
\item Logo que la representa. 
\item ¿Deriva de alguna otra distribución? 
\item ¿Se encuentra activo su desarrollo? ¿De cuándo es la fecha de su última release? 
\item ¿Qué entorno gráfico utiliza?
\item ¿Cuál es su página web oficial?
\item ¿Para qué arquitectura de hardware se encuentra disponible?
\item ¿Proveen software privativo como parte de la instalación? 
\item ¿Cómo están organizadas las aplicaciones? ¿Es sencillo encontrar 
aplicaciones multimedia para cada fin? Por ejemplo, si queremos un editor de sonido.
\end{enumerate}

\subsection*{Aplicaciones multimedia:}
Recorra la distribución y elija aplicaciones multimediales que 
satisfagan las siguientes características: 

\begin{itemize}
\item 4 aplicaciones relacionadas a imágenes estáticas bidimensionales. 
\item 4 aplicaciones relacionadas a sonido
\item 4 aplicaciones relacionadas a video y animación.
\item 3 aplicaciones relacionadas a composición multimedial. 
\end{itemize}

Las aplicaciones elegidas no deben ser obsoletas, es decir deben
tener una comunidad activa y desarrollo no superior a tres años, 
es decir la última versión disponible {\bf por el desarrollador}
no debe superar los tres años. 

Para cada aplicación deberá:
\begin{enumerate}
\item Describir brevemente su funcionalidad. 
\item Características técnicas: Requisitos generales 
de sistema: almacenamiento masivo, RAM, CPU, hardware específico, etc. 
Versión. Licencia. Sitio del desarrollo (homepage/mainstream). 
\item Formatos de datos manipulados. 
\item Dar un ejemplo de uso de la aplicación. ¿Para qué podría servir? 
\item ¿Existe alguna aplicación que sea un estándar de facto y cubra esta
funcionalidad? Por ejemplo el estándar de facto para dibujo CAD suele ser
Autocad (R) de Autodesk, si bien existe FreeCad, Qcad y otros que tienen
funcionalidades similares. 
\item En el caso de aplicaciones de composición: ¿Permiten interactividad? 
¿Son lineales o no lineales? 
\end{enumerate}

\subsection*{Preguntas generales:}
\begin{itemize}
\item ¿Cuál es el modelo de la placa de video? ¿Cómo obtiene esta información?
\item ¿Cuál es el modelo de la placa de sonido? ¿Cómo obtiene esta información?
\item ¿Cuál es el driver utilizado para la tarjeta de video?
\item ¿Cuál es el driver utilizado para la tarjeta de sonido?
\end{itemize}
\end{document}
