%%% LaTeX Template: Article/Thesis/etc. with colored headings and special fonts
%%%
%%% Source: http://www.howtotex.com/

\documentclass[12pt]{article}


\usepackage{apuntes-estilo}
\usepackage{fancyhdr,lastpage}



\def\maketitle{

% Titulo 
 \makeatletter
 {\color{bl} \centering \huge \sc \textbf{
Trabajo práctico N 4 \\
\large \vspace*{-8pt} \color{black} Aplicaciones Multimedia 
 \vspace*{8pt} }\par}
 \makeatother


% Autor
 \makeatletter
 {\centering \small 
	Aplicaciones Libres\\
 	Departamento de Ingeniería de Computadoras \\
 	Facultad de Informática - Universidad Nacional del Comahue \\
 	\vspace{20pt} }
 \makeatother

}

% Custom headers and footers
\fancyhf{} % clear all header and footer fields
\fancypagestyle{plain}{\fancyhf{}}
  	\pagestyle{fancy}
 	\lhead{\footnotesize TP N 2 - Aplicaciones Libres}
 	\rhead{\footnotesize \thepage\ }	% "Page 1 of 2"

\def\ti#1#2{\texttt{#1} & #2 \\ }



\begin{document}

\thispagestyle{empty}
\maketitle
\setlength{\parindent}{0pt}


\section*{Introducción}
El mundo de las aplicaciones multimedia es inmenso. Como administradores de 
sistemas encontraremos cientos de requisitos diferentes sobre imagen, audio, 
video y composición multimedial. 

Este trabajo tendrá varios objetivos con respecto a multimedia:
\begin{itemize} 
\item Conocer terminología básica respecto a cada componente multimedial. 
\item Conocer la problemática básica de las aplicaciones. Cómo se agrupan, cuáles
son las funcionalidades principales. 
\item Visualizar los recursos necesarios a nivel de hardware y software.
\item Vincular aplicaciones con los formatos de archivos manipulados. 
\item Vincular aplicaciones de software libre con aplicaciones utilizadas 
como estándar de facto, posiblemente privativas. 
\end{itemize}

\section*{Descripción}
Es habitual que comunidades de personas con fines comunes aunen esfuerzo para 
resolver ciertos tipo de problemas, el mundo del software, y en particular el 
mundo de los sistemas operativos GNU/Linux no es ajeno a este fenómeno. En 
este caso, y con los objetivos mencionados anteriormente, analizaremos el 
trabajo de comunidades de usuarios y desarrolladores de aplicaciones 
multimediales, que han dado como resultado nuevas distribuciones GNU/Linux.  

Nuesto objetivo no será promover el uso de las distribuciones en sí, sino el 
análisis de las mismas como medio de obtener conocimiento acerca de 
aplicaciones multimedia instaladas en ellas y la organización de las 
mismas (forma en que están agrupadas y presentadas a los usuarios). El 
administrador podrá considerar en el futuro este tipo de soluciones cuando 
se adapte a los requisitos de los usuarios.  

\begin{itemize}
\item Ubuntu Studio \url{http://ubuntustudio.org/}
\item DreamStudio (formerly Dream Studio) \url{http://www.celeum.com/about-dreamstudio/}
\item ArtistX \url{http://artistx.org/blog/}
\item Musix GNU+Linux \url{https://musixdistro.wordpress.com/}
\item AV Linux \url{http://www.bandshed.net/AVLinux.html}
\end{itemize}


Características básicas de la distribución:
\begin{enumerate}
\item País de origen. 
\item Logo que la representa. 
\item ¿Deriva de alguna otra distribución? 
\item ¿Se encuentra activo su desarrollo? ¿De cuándo es la fecha de su última release? 
\item ¿Qué entorno gráfico utiliza?
\item ¿Cuál es su página web oficial?
\item ¿Para qué arquitectura de hardware se encuentra disponible?
\item ¿Proveen software privativo como parte de la instalación? 
\end{enumerate}

Aplicaciones multimedia:


\end{document}
