%%% LaTeX Template: Article/Thesis/etc. with colored headings and special fonts
%%%
%%% Source: http://www.howtotex.com/

\documentclass[12pt]{article}


\usepackage{apuntes-estilo}
\usepackage{fancyhdr,lastpage}
\usepackage{graphicx}



\def\maketitle{

% Titulo 
 \makeatletter
 {\color{bl} \centering \huge \sc \textbf{
Trabajo práctico N 2 \\
\large \vspace*{-8pt} \color{black} Formato interno de archivos
 \vspace*{8pt} }\par}
 \makeatother


% Autor
 \makeatletter
 {\centering \small 
	Aplicaciones Libres\\
 	Departamento de Ingeniería de Computadoras \\
 	Facultad de Informática - Universidad Nacional del Comahue \\
 	\vspace{20pt} }
 \makeatother

}

% Custom headers and footers
\fancyhf{} % clear all header and footer fields
\fancypagestyle{plain}{\fancyhf{}}
  	\pagestyle{fancy}
 	\lhead{\footnotesize TP N 2 - Aplicaciones Libres}
 	\rhead{\footnotesize \thepage\ }	% "Page 1 of 2"

\def\ti#1#2{\texttt{#1} & #2 \\ }



\begin{document}

\thispagestyle{empty}
\maketitle
\setlength{\parindent}{0pt}

\section*{Formato interno}

Elabore una comparativa entre los formatos de archivos propuestos por la 
cátedra. La misma deberá incluir, {\it como mínimo}, la siguiente información:
\begin{itemize} 
\item Breve historia
\item Funcionalidad
\item Formato abierto o cerrado (no publicado).
\item Información sobre licenciamiento/patentes.
\item Aplicaciones que lo manipulan. 
\item Factibilidad de desarrollar un software que lo interprete. 
\item Popularidad. Utilización dentro del país por parte de instituciones pública.  
\item Indique si dichos formatos son reconocidos por el comando file y en 
ese caso cómo es qué los reconoce.   
\item ¿Podría sugerir una comparativa mejor en cuanto a la funcionalidad que proveen estos formatos?
\end{itemize} 

Incluya las referencias que utilice para elaborar la comparativa. 

Ademas del trabajo deben preparar una exposición de media hora con diapositivas
(calcular 2 min. por diapositiva).

Al día de la exposición deberán estar subidos el trabajo y las diapositivas. El
trabajo completo debe ser subido solo por uno de los miembros del grupo, los
demás solo deben subir la caratula.

La organización del trabajo recomendada es la siguiente:

\scalebox{0.39}{

\centering

\fbox{
\begin{minipage}[c][\paperheight]{\paperwidth}

\titlepage

\begin{center}
\ \\
\ \\
\vspace{-1cm}


\ \\

\vspace{0.5cm}
{\Large{\bf \sc Aplicaciones Libres}}\\

\ \\
{\Large { \sc Facultad de Informática}}\\

\ \\
{\Large{\bf \sc Universidad Nacional del Comahue}}\\


\vspace{-2.5cm}
\mbox{\hspace{-1cm}\includegraphics[width=2.5cm,height=2.5cm]{logos/unc.png}\hspace{13cm} \includegraphics[width=2.5cm,height=2.5cm]{logos/fai.png}}


\vspace{6cm}

{\Large {\bf\sc Trabajo Practico: Formato interno de archivo}}\\
\ \\
\ \\
\vspace{3cm}

{\Large Nombre, Apellido Autor1}\\
{\Large Nombre, Apellido Autor2}\\
{\Large Nombre, Apellido Autor3}\\

\vfill
{\Large fecha}\\

\end{center}

\end{minipage}
}
~
\fbox{
\begin{minipage}[c][\paperheight]{\paperwidth}

\section*{Indice:}

\begin{itemize}
	\item Título 1......... Nº de pag.
		\begin{itemize}
			\item Sección 1....... Nº de pag.
			\item Sección 2....... Nº de pag.
		\end{itemize}
	\item Título 2......... Nº de pag.
		\begin{itemize}
			\item Sección 1....... Nº de pag.
			\item Sección 2....... Nº de pag.
		\end{itemize}
\end{itemize}

\section*{Lista de gráficos:}

\begin{itemize}

	\item Título de figura o esquema 1..... Nº de pag.

	\item Título de figura o esquema 2..... Nº de pag.

\end{itemize}

\section*{Introducción}

Introducción al problema, importancia y objetivos.

\section*{Desarrollo:}
\begin{itemize}

	\item Toda información de importancia.

	\item Detallar explicar con vocabulario acorde. 

	\item Citar textos, poner opiniones de personas. Tiene que ser claro y
		preciso, también van a ir las imágenes y/o esquemas.

\end{itemize}

\section*{Conclusión:}

Retomar y analizar lo que se dijo previamente en el desarrollo y demostrar que
se cumplió con el objetivo y/o una conclusión final 

\section*{Bibliografía:}

Material bibliográfico detallado, si es un libro editorial, nombre del libro,
autor,etc. Si es pagina de Internet poner el link y fecha de consulta. 

\end{minipage}
}

}

%Listado de comparativas. 
%\begin{enumerate}
%\item MPEG-3 (mp3) vs. OGG
%\item GIF vs. PNG 
%\item DOCX (MS. Office) vs. ODT
%\item CDR (CorelDraw) vs. SVG 
%\item VSDX (Ms. Visio) vs. DIA (Dia)
%\item PPT (Ms. PowerPoint) vs. ODP 
%\item XLSX (Ms. Excel) vs. gnumeric
%\item MKV (Matroska) vs WMV (Windows Media Video)
%\item RAR vs. gz (GZIP) y 7z
%\item PSD/PDD (Adobe Photoshop Drawing) vs. XCF (GIMP)
%\item BMP (Ms. Windows) vs. PNM (Netpbm)
%\item epub vs. mobi (Mobipocket) 
%\item JSON vs. XML 
%\item PDF vs. PS (PostScript)
%\item FreeCad vs. DWG (Autocad)
%\end{enumerate}
 
\end{document}
