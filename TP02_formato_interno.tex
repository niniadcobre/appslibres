%%% LaTeX Template: Article/Thesis/etc. with colored headings and special fonts
%%%
%%% Source: http://www.howtotex.com/

\documentclass[12pt]{article}


\usepackage{apuntes-estilo}
\usepackage{fancyhdr,lastpage}



\def\maketitle{

% Titulo 
 \makeatletter
 {\color{bl} \centering \huge \sc \textbf{
Trabajo práctico N 2 \\
\large \vspace*{-8pt} \color{black} Formato Interno de archivos
 \vspace*{8pt} }\par}
 \makeatother


% Autor
 \makeatletter
 {\centering \small 
	Aplicaciones Libres\\
 	Departamento de Ingeniería de Computadoras \\
 	Facultad de Informática - Universidad Nacional del Comahue \\
 	\vspace{20pt} }
 \makeatother

}

% Custom headers and footers
\fancyhf{} % clear all header and footer fields
\fancypagestyle{plain}{\fancyhf{}}
  	\pagestyle{fancy}
 	\lhead{\footnotesize TP N 2 - Aplicaciones Libres}
 	\rhead{\footnotesize \thepage\ }	% "Page 1 of 2"

\def\ti#1#2{\texttt{#1} & #2 \\ }



\begin{document}

\thispagestyle{empty}
\maketitle
\setlength{\parindent}{0pt}


\section*{Introducción}
Hemos definido un archivo regular como un flujo de bytes. Este flujo de bytes
tendrá sentido al ser interpretado por determinadas aplicaciones. De este modo, 
si queremos visualizar el contenido de un archivo, digamos pdf, no intentaremos
volcarlo directamente sobre la terminal, por ejemplo a través del comando cat, 
sino que utilizaremos un programa específico que comprenda ese formato, por ejemplo 
evince o xpdf. 

Ahora bien, con el fin de observar mas de cerca ese flujo de bytes, de hecho para 
observar el contenido byte a byte de un archivo particular, podemos utilizar el 
comando od (octal dump). Utilizaremos esta herramienta para adentrarnos en 
la comprensión de los formatos internos de archivos.  

\section*{El caso más sencillo: archivos de texto}
Dado que el formato interno de un archivo de textos es el flujo de bytes más
sencillo de analizar, intentaremos a través de su análisis extraer una serie 
de conclusiones acerca del formato interno de archivos en general y algunas 
cuestiones de interoperabilidad. 

\begin{enumerate}
\item Cree un archivo de texto pequeño (al menos dos líneas) de contenido 
      arbitrario con su editor favorito, guárdelo como archivo1
\item ¿ Qué formato interno indica el comando file sobre archivo1 ? 
\item ¿ Qué aplicaciones podría utilizar para visualizar el contenido de archivo1 ?
\item Observe el contenido de archivo1 utilizando: ``{\tt od -c archivo1}''. ¿Observa
      información adicional a la que vemos a través de otras aplicaciones, como 
      puede ser cat o algún editor de texto? ¿Qué representa esa información?
\item Observe el contenido de archivo1 utilizando: ``{\tt od -b archivo1}''. En este
      caso observamos el contenido numérico de cada byte. Cada grupo de tres
      dígitos octales mostrado formara indica el contenido de un byte. De este modo, 
      podemos observar el flujo de bits que componen un archivo de texto, sin 
      interpretación alguna. 
\end{enumerate}
Ahora cabe preguntarse. ¿Cómo es que un byte almacenado con valor octal 
141 (01100001), al ser visualizado por un editor se ve como el caracter ``a''? 
Esto sucede porque la aplicación que lo muestra hace una interpretación de esos
bits utilizando un código. Observe la salida de ``file archivo1'' nuevamente y
la referencia al sistema de codificación utilizado por el archivo. 
\begin{enumerate}
\item Observando el contenido numérico de cada byte nuevamente ({\tt od -b}), 
compare los número observados con la tabla de caracteres ASCII 
(http://en.wikipedia.org/wiki/ASCII) y corrobore con la salida de {\tt od -c} 
para el mismo archivo.
\item ¿Cómo se llaman y qué función cumplen los caracteres de control observados
en la salidas anteriores? 
\end{enumerate}
      

\end{document}
