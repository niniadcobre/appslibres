%%% LaTeX Template: Article/Thesis/etc. with colored headings and special fonts
%%%
%%% Source: http://www.howtotex.com/

\documentclass[12pt]{article}


\usepackage{apuntes-estilo}
\usepackage{fancyhdr,lastpage}



\def\maketitle{

% Titulo 
 \makeatletter
 {\color{bl} \centering \huge \sc \textbf{
Trabajo práctico N 2 \\
\large \vspace*{-8pt} \color{black} Formato Interno de archivos
 \vspace*{8pt} }\par}
 \makeatother


% Autor
 \makeatletter
 {\centering \small 
	Aplicaciones Libres\\
 	Departamento de Ingeniería de Computadoras \\
 	Facultad de Informática - Universidad Nacional del Comahue \\
 	\vspace{20pt} }
 \makeatother

}

% Custom headers and footers
\fancyhf{} % clear all header and footer fields
\fancypagestyle{plain}{\fancyhf{}}
  	\pagestyle{fancy}
 	\lhead{\footnotesize TP N 2 - Aplicaciones Libres}
 	\rhead{\footnotesize \thepage\ }	% "Page 1 of 2"

\def\ti#1#2{\texttt{#1} & #2 \\ }



\begin{document}

\thispagestyle{empty}
\maketitle
\setlength{\parindent}{0pt}


\section*{Introducción}
Hemos definido un archivo regular como un flujo de bytes. Este flujo de bytes
tendrá sentido al ser interpretado por determinadas aplicaciones. De este modo, 
si queremos visualizar el contenido de un archivo, digamos pdf, no intentaremos
volcarlo directamente sobre la terminal, por ejemplo a través del comando cat, 
sino que utilizaremos un programa específico que comprenda ese formato, por ejemplo 
evince o xpdf. 

Ahora bien, con el fin de observar mas de cerca ese flujo de bytes, de hecho para 
observar el contenido byte a byte de un archivo particular, podemos utilizar el 
comando od (octal dump). Utilizaremos esta herramienta para adentrarnos en 
la comprensión de los formatos internos de archivos.  

\section*{El caso más sencillo: archivos de texto}
Dado que el formato interno de un archivo de textos es el flujo de bytes más
sencillo de analizar, intentaremos a través de su análisis extraer una serie 
de conclusiones acerca del formato interno de archivos en general y algunas 
cuestiones de interoperabilidad. 

\begin{enumerate}
\item En un sistema GNU/Linux, cree un archivo de texto pequeño (al menos dos líneas) 
      de contenido arbitrario con su editor favorito, guárdelo como {\tt archivo1}.
\item ¿ Qué formato interno indica el comando {\tt file} sobre archivo1 ? 
\item ¿ Qué aplicaciones podría utilizar para visualizar el contenido de archivo1 ?
\item Observe el contenido de archivo1 utilizando: ``{\tt od -c archivo1}''. ¿Observa
      información adicional a la que vemos a través de otras aplicaciones, como 
      puede ser cat o algún editor de texto? ¿Qué representa esa información?
\item Observe el contenido de archivo1 utilizando: ``{\tt od -b archivo1}''. En este
      caso observamos el contenido numérico de cada byte. Cada grupo de tres
      dígitos octales mostrado, indica el contenido de un byte. De este modo, 
      podemos observar el flujo de bits que componen un archivo de texto.
\end{enumerate}
Ahora cabe preguntarse. ¿Cómo es que un byte almacenado con valor octal 
141 (01100001), al ser visualizado por un editor se ve como el caracter ``a''? 
Esto sucede porque la aplicación que lo muestra hace una interpretación de esos
bits utilizando un código. Observe la salida de ``{\tt file archivo1}'' nuevamente y
la referencia al sistema de codificación utilizado por el archivo. 
\begin{enumerate}
\item Observando el contenido numérico de cada byte nuevamente ({\tt od -b archivo1}), 
compare dichos valores con la tabla de caracteres ASCII 
(http://en.wikipedia.org/wiki/ASCII) y corrobore el mapeo con la salida de {\tt od -c archivo1}  
para el mismo archivo.
\item ¿Cómo se llaman y qué función cumplen el/los caracter/es de control observados
en la salidas anteriores? 
\end{enumerate}

Hasta aquí no hemos enfrentado ningún problema de interoperabilidad con otros 
sistemas operativos. En principio el formato interno de archivos de texto no 
pareciera representar ningún desafío a la hora de ser interpretados en aplicaciones
de uno u otro sistema operativo, sin embargo el uso de los caracteres de control 
que representan el fin de línea es el comienzo de la discordia. 

En los sistemas GNU/Linux el caracter de control que indica el fin de una línea, en 
un archivo de texto es LF (Line Feed), visualizado como ``\textbackslash n''
\footnote{http://en.wikipedia.org/wiki/ASCII\#ASCII\_control\_code\_chart}. Mientras
que en los sistemas Windows (C), será la combinación de CR (Carriage Return) \textbackslash r, y 
LF (Line Feed) lo que determine el fin de una línea. 

Vemos más de cerca esta ``insignificante'' discrepancia, y analicemos las implicaciones
para los usuarios normales. 

\begin{enumerate}
\item Abra el archivo1 con el editor Vim y, en modo normal, ejecute: {\tt :set ff?}. Este 
comando nos mostrará qué tipo de {\it fin de línea} se está utilizando para leer el archivo1. 
Las opciones pueden ser unix, dos o mac (algunas versiones del sistema operativo Mac OS
utilizan sólo CR (Carriage Return) para finalizar una línea). ``ff'' se corresponde con 
``fileformat''.
	\begin{itemize}
	\item Vuelva a observar con {\tt od -c} el caracter de control de fin de línea de archivo1 y
	verifique lo mostrado por Vim. 
	\item Vuelva a observar qué formato de archivo indica el comando {\tt file} sobre 
	archivo1. 
	\end{itemize} 
\item Abra el archivo1 con el editor Vim y en modo normal, ejecute: {\tt :bufdo! set ff=dos|w} 
\footnote{http://Vim.wikia.com/wiki/File\_format}. 
	\begin{itemize}
	\item Consulte nuevamente el formato en uso: {\tt :set ff?}. ¿Cambió?
	\item Vuelva a observar con {\tt od -c} el caracter de control de fin de línea de archivo1 y
	verifique lo mostrado por Vim. 
	\item Vuelva a observar qué formato de archivo indica el comando {\tt file} sobre 
	archivo1. ¿Cambió algo con respecto a la versión original? 
	\item ¿Qué función cumple entonces el comando {\tt :bufdo!} utilizado al comienzo de este 
	ejercicio?
	\end{itemize} 
\end{enumerate}

Teniendo en cuenta todo lo visto hasta el momento, cabe ahora preguntarnos: 

\begin{enumerate}
	\item ¿Cómo afecta esta discrepancia a los usuarios normales? Imagine un usuario de un 
	sistema GNU/Linux que manda un correo electrónico con un archivo de texto adjunto a un usuario de un 
	sistema Microsoft Windows (C). ¿Qué sucederá cuando este último quiera abrir el archivo 
	de texto, digamos con Notepad?
	\item Observe editores gráficos como gedit. ¿Qué opciones relativas a estas cuestiones de 
	interoperabilidad poseen a la hora de guardar un archivo? ¿Cuál sería el equivalente en  
	gedit a la sentencia {\tt :bufdo! set ff=dos|w } de Vim? 
\end{enumerate}
     
Este es el ejemplo más sencillo que podemos analizar acerca de cómo distintas aplicaciones y 
distintos sistemas operativos pueden hacer interpretaciones diferentes del formato interno de 
un archivo. Imagine ahora que en lugar de un archivo de texto, hablamos de un archivo con 
formato enriquecido como puede ser el generado por Microsoft Office, o LibreOffice. ¿Cuántas
discrepancias podrán surgir en la interpretación del formato? Parte de los objetivos de 
esta asignatura será vislumbrar esta clase de inconvenientes, cómo afecta esto a los usuarios y
qué podemos hacer los administradores para ayudarlos. 

\section*{Formato interno}

Elabore una comparativa entre los formatos de archivos propuestos por la 
cátedra. La misma deberá incluir, {\it como mínimo}, la siguiente información:
\begin{itemize} 
\item Breve historia
\item Funcionalidad
\item Formato abierto o cerrado (no publicado).
\item Información sobre licenciamiento/patentes.
\item Aplicaciones que lo manipulan. 
\item Factibilidad de desarrollar un software que lo interprete. 
\item Popularidad. Utilización dentro del país por parte de instituciones pública.  
\item Indique si dichos formatos son reconocidos por el comando file y en 
ese caso cómo es qué los reconoce.   
\item ¿Podría sugerir una comparativa mejor en cuanto a la funcionalidad que proveen estos formatos?
\end{itemize} 
Incluya las referencias que utilice para elaborar la comparativa. 

Dicha comparativa deberá documentarse en el sitio 
http://asimov.fi.uncoma.edu.ar/dokuwiki, en la página 
para ``Aplicaciones libres 2014''. 

Listado de comparativas. 
\begin{enumerate}
\item MPEG-3 (mp3) vs. OGG
\item GIF vs. PNG 
\item DOCX (MS. Office) vs. ODT
\item CDR (CorelDraw) vs. SVG 
\item VSDX (Ms. Visio) vs. DIA (Dia)
\item PPT (Ms. PowerPoint) vs. ODP 
\item XLSX (Ms. Excel) vs. gnumeric
\item MKV (Matroska) vs WMV (Windows Media Video)
\item RAR vs. gz (GZIP) 
\item PSD/PDD (Adobe Photoshop Drawing) vs. XCF (GIMP)
\item BMP (Ms. Windows) vs. PNM (Netpbm)
\item epub vs. mobi (Mobipocket) 
\item JSON vs. XML 
\item PDF vs. PS (PostScript)
\item FreeCad vs. DWG (Autocad)
\end{enumerate}
 
\end{document}
