%%% LaTeX Template: Article/Thesis/etc. with colored headings and special fonts
%%%
%%% Source: http://www.howtotex.com/

\documentclass[12pt]{article}


\usepackage{apuntes-estilo}
\usepackage{fancyhdr,lastpage}
\usepackage{color,colortbl}
\usepackage{verbatim}

\def\maketitle{

% Titulo 
 \makeatletter
 {\color{bl} \centering \huge \sc \textbf{
  Aplicaciones Multimedia\\ 
\large \vspace*{-8pt} \color{black}Introducción a aplicaciones multimedia. 
 \vspace*{8pt} }\par}
 \makeatother

% Autor
\makeatletter
 {\centering \small 
 	Departamento de Ingeniería de Computadoras \\
 	Facultad de Informática - Universidad Nacional del Comahue \\
 	\vspace{20pt} }
 \makeatother

}

% Custom headers and footers
\fancyhf{} % clear all header and footer fields
\fancypagestyle{plain}{\fancyhf{}}
  	\pagestyle{fancy}
 	\lhead{\footnotesize Aplicaciones multimedia - Departamento de Ingeniería de Computadoras}
 	\rhead{\footnotesize \thepage\ }	% ''Page 1 of 2''

\def\ti#1#2{\texttt{#1} & #2 \\ }



\begin{document}

\thispagestyle{empty}
\maketitle
\setlength{\parindent}{0pt}

\section*{Introducción}


Para comprender el alcance de la palabra multimedia analizaremos distintas 
definiciones. Según la Real Academia Española, multimedia se refiere a un 
adjetivo que indica: ``que utiliza conjunta y simultáneamente diversos 
medios, como imágenes, sonidos y texto, en la transmisión de una 
información.''\cite{raemm}

\begin{center}
\includegraphics{multimedia.png}
\end{center}

Desde un punto de vista etimológico podemos decir que multimedia se 
refiere a: multi y media. Tal que multi se refiere a  muchos, y media a: 
substancia intermedia a través de la cual algo es transmitido o 
transportado. Un medio de comunicación masivo, tal como un periódico o 
la televisión.\cite{ramyer}

Estas definiciones nos das un primer acercamiento al alcance de la palabra
multimedia. Sin embargo, si nos preguntamos qué aspectos de este alcance 
serían importantes cubrir desde el punto de vista de {\it aplicaciones 
multimediales}, o desde la realidad de un administrador de sistemas y un 
entorno de trabajo convencional, como puede ser una oficina, todavía queda
mucho por aclarar. 

En este sentido la definición de Franklin Kuo, establece dos aspectos del
concepto de multimedia en los cuales haremos foco:

\begin{itemize}
\item Multimedia se refiere a la representación de medios mixtos de información
-texto, datos, imágenes, audio y video- como señales digitales. 
\item Comunicaciones multimedia, se refiere a la tecnología requerida para 
manipular, transmitir y controlar estas señales audiovisuales a través de un 
canal de comunicación.\cite{frankuo}
\end{itemize}

Teniendo en cuenta estos dos aspectos, si nos concentramos en contenidos multimedia 
dentro computadora aislada, es decir,  sin conexión a red (ya sea LAN o 
Internet), entonces nos enfocaremos por un lado en la representación y almacenamiento 
de datos multimediales, utilizando para ello los conocimientos previos 
adquiridos sobre formato interno de archivos regulares. Por otro lado, en 
las {\it aplicaciones multimedia} que nos permiten {\it generar} y manipular 
dichos contenidos localmente. 

Sin embargo, dado la expansión de Internet, y el frecuente desarrollo 
de redes de área local (LAN), debemos considerar el segundo aspecto mencionado
por Kuo referente a las {\it Comunicaciones multimediales}. La transmisión 
de datos multimediales sin dudas representa un desafío para los administradores
de sistemas en cuánto al análisis de recursos utilizados en dicha transmisión.
Una transmisión de video en vivo que no llega a tiempo a sus destinos, será 
una comunicación multimedial no exitosa, por lo que el análisis de los 
recursos necesarios se vuelve una tarea compleja e imprescindible en los 
sistemas modernos. 

Por último y nos menos importante, abordaremos brevemente los aspectos legales 
sobre la generación de contenidos multimedia. 

\section*{Tipos de medios}
\subsection*{Estáticos, medio discreto independiente del tiempo}
\subsection*{Dinámicos, medio continuo dependiente del tiempo}

\section*{Transmisión}


\section*{Aspectos Legales}

\begin{thebibliography}{1}

\bibitem{raemm}
Real Academia Española, \emph{Multimedia}.\hskip 1em plus
0.5em minus 0.4em\url{http://lema.rae.es/drae/?val=multimedia}

\bibitem{ramyer}
Ramesh Yerraballi, \emph{Multimedia Systems, Concepts Standards and 
Practice}.\hskip 1em plus  0.5em minus 0.4em
\url{http://users.ece.utexas.edu/~ryerraballi/MSB/Contents.html}

\bibitem{frankuo}
Franklin F. Kuo, J. Joaquin Garcia Luna-Aceves y Wolfgang Effelsberg. 
\emph{Multimedia Communications: Protocols and Applications}.\hskip 1em plus
  0.5em minus 0.4em\relax Publisher: Prentice Hall; ISBN-13: 978-0138569235  

\end{thebibliography}{1}

\section*{Licencia}
Copyright (C) 2014 Lechner Miriam.

Se concede autorización para copiar, distribuir y/o modificar este documento
bajo los términos de la Licencia Creative Commons Atribución-CompartirDerivadasIgual 3.0 Unported. 

\end{document}
