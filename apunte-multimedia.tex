	%%% LaTeX Template: Article/Thesis/etc. with colored headings and special fonts
%%%
%%% Source: http://www.howtotex.com/

\documentclass[12pt]{article}


\usepackage{apuntes-estilo}
\usepackage{fancyhdr,lastpage}
\usepackage{color,colortbl}
\usepackage{verbatim}

\def\maketitle{

% Titulo 
 \makeatletter
 {\color{bl} \centering \huge \sc \textbf{
  Aplicaciones Multimedia\\ 
\large \vspace*{-8pt} \color{black}Introducción a aplicaciones multimedia. 
 \vspace*{8pt} }\par}
 \makeatother

% Autor
\makeatletter
 {\centering \small 
 	Departamento de Ingeniería de Computadoras \\
 	Facultad de Informática - Universidad Nacional del Comahue \\
 	\vspace{20pt} }
 \makeatother

}

% Custom headers and footers
\fancyhf{} % clear all header and footer fields
\fancypagestyle{plain}{\fancyhf{}}
  	\pagestyle{fancy}
 	\lhead{\footnotesize Aplicaciones multimedia - Departamento de Ingeniería de Computadoras}
 	\rhead{\footnotesize \thepage\ }	% ''Page 1 of 2''

\def\ti#1#2{\texttt{#1} & #2 \\ }



\begin{document}

\thispagestyle{empty}
\maketitle
\setlength{\parindent}{0pt}

\section*{Introducción}

Para comprender el alcance de la palabra multimedia analizaremos distintas 
definiciones. Según la Real Academia Española, multimedia se refiere a un 
adjetivo que indica: ``que utiliza conjunta y simultáneamente diversos 
medios, como imágenes, sonidos y texto, en la transmisión de una 
información.''\cite{raemm}


\begin{figure}[h]
\centering
\includegraphics[width=0.8\textwidth]{multimedia.png}
\renewcommand{\figurename}{Fig.}
\caption{https://openclipart.org/detail/201582/multimedia-by-star4clover-201582}
\label{contexto:figura}
\end{figure}

Desde un punto de vista etimológico podemos decir que multimedia se 
refiere a: multi y media. En este sentido ``multi'' se refiere a  muchos, y ``media'' a la
sustancia intermedia a través de la cual algo es transmitido o 
transportado; esto podría ser un medio de comunicación masivo, tal como un periódico o 
la televisión.\cite{ramyer}

Estas definiciones nos dan un primer acercamiento al alcance de la palabra
multimedia. Sin embargo, si nos preguntamos qué aspectos de este alcance 
serían importantes cubrir desde el punto de vista de {\it aplicaciones 
multimedia}, o desde la realidad de un administrador de sistemas y un 
entorno de trabajo convencional, como puede ser una oficina, todavía queda
mucho por aclarar. 

En este sentido la definición de Franklin Kuo, establece dos aspectos del
concepto de multimedia en los cuales haremos foco:

\begin{itemize}
\item Multimedia se refiere a la representación de medios mixtos de información
-texto, datos, imágenes, audio y video- como señales digitales. 
\item Comunicaciones multimedia, se refiere a la tecnología requerida para 
manipular, transmitir y controlar estas señales audiovisuales a través de un 
canal de comunicación.\cite{frankuo}
\end{itemize}


Teniendo los dos aspectos mencionados por Kuo, si nos concentramos en contenidos 
multimedia dentro de una computadora aislada, es decir, sin conexión a red (ya sea LAN o 
Internet), entonces nos enfocaremos por un lado en la representación y almacenamiento 
de datos multimedia, utilizando para ello los conocimientos previos 
adquiridos sobre formato interno de archivos regulares y por otro lado, en 
las {\it aplicaciones multimedia} que nos permiten {\it generar} y manipular 
dichos contenidos localmente. 

Sin embargo, dado la expansión de Internet, y el frecuente desarrollo 
de redes de área local (LAN), debemos dar consideración al segundo aspecto mencionado
por Kuo referente a las {\it Comunicaciones multimedia}. La transmisión 
de datos multimedia sin dudas representa un desafío para los administradores
de sistemas en cuánto al análisis de recursos utilizados en dicha transmisión.
Una transmisión de video en vivo que no llega a tiempo a sus destinos, será 
una comunicación multimedia no exitosa, por lo que el análisis de los 
recursos necesarios se vuelve una tarea compleja e imprescindible en los 
sistemas modernos. 

Por último y no menos importante, abordaremos brevemente los aspectos legales 
sobre la generación y transmisión de contenidos multimedia. En muchos casos,
hemos naturalizado el uso de imágenes, sonidos, textos y videos obtenidos 
desde Internet, sin prácticamente cuestionarnos acerca de la validez 
legal del uso que hacemos. Quizá es más evidente cuando se trata de 
películas o audio, ya que existen multiplicidad de campañas que, mal o 
bien, intentan introducir las consideraciones legales acerca de la descarga y 
reproducción de estos contenidos. Sin embargo, cuántos de nosotros nos 
preguntamos, por ejemplo al desarrollar el contenido de una wiki, si la 
imagen que acabamos de incorporar puede ocasionar un problema legal. 

En resumen, abordaremos multimedia siguiendo tres pilares:
\begin{itemize}
\item Aspectos estáticos: creación, reproducción (aplicaciones y formatos) y 
consumo de recursos, como ser almacenamiento y requisitos de procesamiento. 
\item Aspectos dinámicos: transmisión dentro de redes de área local (LAN) e 
Internet. 
\item Aspectos legales.  
\end{itemize}

\section*{Usos}

Los diferentes formatos de multimedia analógica o digital tienen la 
intención de mejorar la experiencia de los usuarios, por ejemplo para 
que la comunicación de la información sea más fácil y rápida, o, en 
el entretenimiento y el arte, para trascender la experiencia común.\cite{wikipmmes} 

Los usos de la multimedia digital son potencialmente ilimitados. Algunos 
ejemplos que podríamos mencionar son:  

\begin{itemize}
\item {\bf Uso comercial}: publicidad, artistas que distribuyen sus obras como contenidos 
multimedia digitales, venta de productos a través de catálogos en la web, capacitaciones online. 
\item {\bf Entretenimiento y bellas artes}: nuevos desarrollos artísticos y nuevas formas 
de presentación del arte existente, como las galerías de arte online (Google Art Project, 
Museo del Prado, etc) han cobrado vida en los últimos años.  
\item {\bf Educación}: PEDCO podría ser un ejemplo, enciclopedias interactivas, entrenamientos
interactivos, etc. 
\item {\bf Periodismo}: La explosión de los periódicos digitales, la incorporación de dispositivos 
multimedia en los noticieros televisivos, son algunos ejemplos. 
\item {\bf Ingenierías, arquitectura, ciencias en general, medicina}: todas las ciencias se ven 
beneficiadas por el uso de multimedia. La simulación y el modelado son un recurso 
valioso en todas estas áreas del conocimiento. 
\end{itemize}


\section*{Categorizaciones}

Desarrollaremos brevemente dos categorizaciones, ambas relacionadas a la 
percepción del usuario con respecto al contenido multimedia. En un caso 
nos referiremos a la interactividad del usuario, mientras que en el otro 
haremos referencia al componente temporal que puede o no afectar a la
reproducción de los contenidos. 

A la hora de evaluar contenidos multimedia, tendremos en cuenta entonces
por un lado los mecanismos de interacción (si los hubiere), y por otro 
cómo afecta el tiempo a la reproducción de elementos individuales. 

\subsection*{Interactividad}
En cuanto a la interacción que tiene el usuario con un contenido 
multimedia (como un todo, es decir un conjunto de elementos individuales
como texto, imagen, etc), podemos dividir dichos contenidos en: 
\begin{itemize}
\item Lineal: la reproducción de este tipo de contenidos progresa 
sin intervención del observador. Por ejemplo, una presentación cinematográfica.  
\item No lineal: en este caso, la interactividad es utilizada para 
controlar el progreso de la reproducción del contenido multimedia. Por 
ejemplo un video juego, una enciclopedia virtual en donde el observador
avanza a través de los contenidos que prefiere (ejemplo de hypermedia, 
wikipedia es una implementación posible), etc. 
\end{itemize}

\subsection*{Tipos de medios}
Por otra parte, teniendo en cuenta los elementos individuales que componen un 
contenido multimedia, podemos distinguir dos grandes tipos de medios: 

\begin{itemize}
\item{Estáticos, medio discreto independiente del tiempo}: 
Dentro de esta clase encontramos a las imágenes y gráficos fijos, y texto. La información 
en este medio consiste exclusivamente de una secuencia de elementos individuales 
sin un componente de tiempo.
\item{Dinámicos, medio continuo dependiente del tiempo}: 
Aquí encontramos al sonido, animaciones y video. La información no sólo es 
expresada por su valor individual, sino que también por el momento 
de su ocurrencia.
\end{itemize}

Claramente los medios continuos representarán un mayor desafío para el 
administrador a la hora de la transmisión a través de una red. En este 
sentido también podemos subdividir la clasificación en {\it transmisión
en vivo o grabada}. En ambos casos es posible la interactividad, pero 
la transmisión en vivo impone aún más restricciones y desafíos para 
considerarse exitosa. 

Una aclaración importante acerca de esta clasificación es que la noción 
de {\it dependencia del tiempo} no tiene relación con la representación
del dato, en el sentido de que ambos medios, discreto y continuo, en su 
representación física puede no tener información acerca del componente 
temporal. El aspecto temporal en este caso, es más bien la percepción del 
que escucha u observa.\cite{ramyer}  

\section*{Aplicaciones multimedia}

Podríamos comenzar esta sección realizando una lista de aplicaciones 
libres conocidas para multimedia y hacer una descripción breve de cada
una de ellas. Sin embargo además de caer en la obsolescencia mientras 
leemos este apunte, sería imposible ser exhaustivos. Utilizaremos 
algunas aplicaciones a modo de ejemplo y, que al momento de escribir este 
apunte gozan de cierta popularidad.  

El objetivo será entonces, caracterizar las aplicaciones pensando
siempre en nuestro rol de administradores de sistema y, en particular 
en el asesoramiento que usualmente nos es demandado por los usuarios. No
será nuestro objetivo ahondar en el diseño multimedia, sino obtener 
los conocimientos básicos para poder entender la problemática, sugerir 
posibles soluciones o simplemente ser el punto de partida para un análisis 
más profundo cuando el entorno a administrar así lo demande. 

En cada caso trataremos de identificar: 
\begin{itemize}
\item Funcionalidad: ¿Sirve para producir elementos individuales (discretos o continuos) o 
bien es una herramienta que permite la vinculación y presentación de elementos individuales? 
\item Consumo de recursos. Necesidades de la aplicación con respecto al 
hardware y al sistema operativo. Por ejemplo ciertas aplicaciones para generar 
elementos de animación requieren hardware gráfico específico; otras pueden demandar 
una conexión de red para transmitir el contenido, etc. 
\end{itemize}

\subsection*{Gráficos e imágenes estáticas}

Una problemática frecuente es la necesidad de generar y manipular gráficos e imágenes 
bidimensionales. Por citar algunos de los infinitos requisitos posibles, un usuario 
podría necesitar: 
\begin{itemize}
	\item Editar fotos. El usuario desea retocar una o cientos de imágenes capturadas
	      con alguna cámara digital, o escaneadas. En este caso no sólo debemos pensar en el problema de 
	      encontrar software que le permite realizar la edición, sino que también debemos
	      pensar en el mecanismo que le permita importar las fotos como imágenes digitales
	      dentro de su computadora para su posterior edición.    
	\item Creación de imágenes matriciales e imágenes matriciales multicapa (como las generadas por 
	Adobe Photoshop o Gimp).  
	\item Cambiar el formato de una imagen. Por ejemplo pasar uno o cientos de archivos de 
	un formato PNG a JPG. Además podría requerir asesoramiento de qué formato sería el 
	indicado si tuviese que visualizarlo en otros sistemas operativos. 
	\item Crear un diagrama vectorial. Por ejemplo el usuario desea crear un mapa conceptual, o
	un organigrama jerárquico que represente a la organización en la que trabaja.    
	\item Crear una gráfico vectorial (como caso general de lo anterior). Por ejemplo el 
	usuario podría querer crear un logo para la empresa que pueda visualizarse luego en 
	una página web.  
	\item Tomar una captura de pantalla y editarla para luego incorporarla en una presentación. 
	\item Controlar una cámara digital desde la computadora. 
\end{itemize}

Deberemos tener presente al menos que existen: {\bf{\it imágenes matriciales e imágenes vectoriales}}. 
En este sentido:  

``Una {\it imagen vectorial} es una imagen digital formada por objetos geométricos independientes (segmentos, 
polígonos, arcos, etc.), cada uno de ellos definido por distintos atributos matemáticos de forma, de 
posición, de color, etc. Por ejemplo un círculo de color rojo quedaría definido por la posición 
de su centro, su radio, el grosor de línea y su color.

Este formato de imagen es completamente distinto al formato de las imágenes de mapa de bits, también 
llamados imágenes matriciales, que están formados por píxeles. El interés principal de los gráficos 
vectoriales es poder ampliar el tamaño de una imagen a voluntad sin sufrir la pérdida 
de calidad que sufren los mapas de bits. De la misma forma, permiten mover, estirar y 
retorcer imágenes de manera relativamente sencilla.'' \cite{wikiives}

``A las {\it imágenes de mapa de bits} se las suele definir por su altura y anchura (en píxeles) y por 
su profundidad de color (en bits por píxel), que determina el número de colores distintos que se 
pueden almacenar en cada punto individual, y por lo tanto, en gran medida, la calidad del color de la imagen.''
\cite{wikiimes}

A esto podemos adicionar la existencia de formatos de imagen mas complejos que involucran las antes mencionadas
como base, como ser XCF de Gimp, o PSD de Adobe Photoshop. 

Analizando un poco el consumo de recursos, en lo que a formatos respecta, en general una 
imagen matricial ocupa mayor espacio de almacenamiento que su equivalente vectorial. En el 
primer caso, el tamaño es proporcional al tamaño de la matriz de píxeles (resolución) y a la
profundidad de color de cada píxel (bits por píxel), mientras que en el segundo caso estamos 
hablando de representaciones matemáticas que no varían con el tamaño visualizado. Si bien puede 
darse el caso opuesto, en que la complejidad de las fórmulas matemáticas descriptivas superen 
la versión de mapa de bits, la percepción general es la primera. 

Pensando en las aplicaciones que manipulan estos formatos, en general, la mayoría demandarán al 
menos: hardware gráfico (vulgarmente una placa de video); 
monitor para visualización; cierta capacidad de procesamiento (CPU) y memoria principal (RAM), ya que 
las aplicaciones gráficas, aún cuando nos referimos a imágenes bidimensionales, suelen consumir bastantes 
recursos. También podríamos necesitar elementos de captura de imágenes: cámaras digitales, escáneres, etc.  
Desde los requisitos del sistema operativo, si bien cierto tipo de procesamiento en masa,
por ejemplo el escalado de cientos de imágenes, podría realizarse sin la necesidad de un
entorno gráfico (por ejemplo utilizando el programa convert), en la mayoría de los casos será una 
necesidad. Por otro lado, será necesario que el sistema operativo cuente con los controladores
apropiados para el correcto funcionamiento del hardware, lo cual no es una tarea trivial en 
el mundo del software libre, ya que muchos fabricantes de placas de video (y podríamos decir de hardware
en general) no proveen controladores libres y por ende, puede que las versiones libres 
disponibles (creadas por la comunidad) no exploten todas las características del hardware en cuestión.  

Algunas aplicaciones libres muy populares que podemos mencionar en esta sección son: 
\begin{itemize}
\item Gimp \url{https://directory.fsf.org/wiki/GIMP}, que permite 
trabajar con mapas de bits, e imágenes multicapa. 
\item Inkscape \url{https://directory.fsf.org/wiki/Inkscape}, 
para imágenes vectoriales. 
\item Dia \url{https://directory.fsf.org/wiki/Dia} y Graphviz 
\url{https://directory.fsf.org/wiki/Graphviz}, para diagramas vectoriales de 
varios tipos (mapas mentales, grafos, redes, etc). 
\item DigiKam \url{https://directory.fsf.org/wiki/DigiKam}, nos permite la descarga y organización de 
imágenes tomadas con cámaras digitales. 
\item El comando {\tt convert} realiza varios tipos de procesamiento
en lote (escalado, conversión de formatos, etc). 
\item Hugin \url{https://directory.fsf.org/wiki/Hugin},  
permite la creación de panorámicas a partir de fotos individuales.  
\item Entangle \url{http://entangle-photo.org/}, que permite controlar una cámara 
digital desde la computadora.
\end{itemize}


\subsection*{Sonido}

El sonido es una onda continua que viaja a través del aire y que es percibida por el 
oído humano debido a los cambios de presión generados por esta onda sobre el mismo. 
La representación digital de esas ondas y los recursos necesarios para almacenar, 
reproducir y manipular estos sonidos serán de interés en esta sección (en inglés ``digitize''). . 

Uno de los aspectos a considerar será la representación digital de estas ondas sonoras.
En sí la resolución de estas ondas es infinita. Deberemos entonces pasar de una 
señal analógica continua a alguna forma de representación discreta, posible de ser  
almacenada numéricamente en una computadora. Este será el proceso de digitalización. 
Por ejemplo al utilizar un grabador de sonido (ej. gnome-sound-recorder) proveniente 
de un micrófono y guardar un archivo con esa información, estamos digitalizando la 
onda sonora analógica proveniente del micrófono. 

Sin entrar en los detalles de muestreo y cuantificación de las ondas analógicas, 
una onda sonora será representada digitalmente por una secuencia 
de muestras tomadas en el tiempo en distintos puntos de la onda. Cada muestra 
tendrá un valor que representa la amplitud de la onda en ese punto del tiempo.
La cantidad de muestras por segundo (unidad Hz) y la cantidad de bits usados 
para la amplitud de onda (cuantificación) determinan la calidad de la 
digitalización. Por ejemplo para un CD de audio estándar se utilizan 44100Hz
(44100 muestras por segundo) con una cuantificación de la amplitud de 16 bits
(65536 posibles valores para la amplitud). 


\begin{figure}[h]
\centering
\includegraphics[width=0.8\textwidth]{Sampling.png}
\renewcommand{\figurename}{Fig.}
\caption{http://commons.wikimedia.org/wiki/File:Signal\_Sampling.png}
\label{contexto:figura}
\end{figure}

Una primera deducción que podemos hacer en cuanto al consumo de recursos, y
teniendo en cuenta el muestreo y cuantificación, es que a mayor cantidad de 
muestras por segundo (Hz) y mayor cantidad de bits utilizadas para la
representación de la amplitud, mayor será el tamaño del archivo digital 
generado. A su vez, esto será inversamente proporcional a lo que sucede 
con la calidad de la representación. 

En este sentido, también debemos tener en cuenta que ciertos formatos 
de archivo de audio ocuparán más o menos espacio de almacenamiento, dependiendo 
de estos parámetros y del posible uso de algoritmos de compresión. Por ejemplo 
un archivo con formato {\it wav}\footnote{http://es.wikipedia.org/wiki/Waveform\_Audio\_Format}   
usualmente sin compresión, ocupará mayor espacio de almacenamiento que su
equivalente {\it Ogg Vorbis}\footnote{http://www.vorbis.com/} con 
compresión.

Si quisiéramos ahondar en el tema de sonido y las conversiones, deberíamos 
incluir en nuestro análisis el proceso inverso a la digitalización, por el 
cual un archivo digital de sonido termina produciendo una onda sonora 
analógica. En este aspecto sólo abordaremos las necesidades de hardware
para que esto suceda y las aplicaciones que permiten la reproducción. La
conformación de ondas analógicas a partir de archivos de audio digital 
queda fuera del alcance de este curso. 

Pensemos ahora potenciales requisitos que pueden surgir en cualquier entorno 
de trabajo, relacionados a sonido:
\begin{itemize}
\item Reproducir archivos de audio de distintos tipos y formatos.  
\item Editar un archivo de audio: recortar secciones, 
agregar efectos de sonido, pausas, etc. 
\item Reconocimiento de voz. 
\item Cambios de formato: compresión de audio. 
\item Conferencia de audio. 
\item Controlar la tarjeta de sonido: volumen, efectos, etc.
\item Captura de sonido: micrófono, instrumentos, etc. 
\end{itemize}

Los recursos de hardware necesarios dependerán de estos requisitos.  
En general necesitaremos al menos una tarjeta de sonido que pueda 
recibir y producir señales de audio; las mismas podrán ser integradas
en la placa base o externas, generalmente conectadas a algún bus de
expansión (PCI, PCIE, etc.). Desde el sistema operativo deberemos ser 
capaces de reconocer la presencia de dicho hardware, así como también 
los controladores del sistema operativo adecuados para el efectivo 
funcionamiento del mismo.


\begin{figure}[h]
\centering
\includegraphics[width=0.8\textwidth]{soundcards.png}
\renewcommand{\figurename}{Fig.}
\caption{http://en.wikipedia.org/wiki/File:Turtle\_Beach\_Sound\_Card\_\%28Catalina\%29.png}
\label{contexto:figura}
\end{figure}

Las tarjetas de sonidos serán las encargadas del proceso de conversión 
analógica/digital y viceversa, para captura/reproducción de audio, explicado 
anteriormente. Adicionalmente, elementos analógicos como ser: parlantes y 
micrófono serán necesarios según el caso. Dicho elementos se conectan 
a la tarjeta de sonido a través de diferentes conectores {\it minijack}.  

En cuanto a {\it aplicaciones libres} que permitan satisfacer diferentes
requisitos relacionados a audio, podemos decir que la mayoría de las distribuciones
ofrecen aplicaciones multimedia para reproducción, mencionaremos algunos 
ejemplos:

\begin{itemize}
\item Rhythmbox \url{https://directory.fsf.org/wiki/Rhythmbox}. GNOME. 
\item Amarok. \url{https://directory.fsf.org/wiki/Amarok}. KDE 
\item Clementine. \url{https://directory.fsf.org/wiki/Clementine}. 
Basado en Amarok. 
\item XMMS. \url{https://directory.fsf.org/wiki/Xmms}. 
\item VLC Media Player. \url{https://directory.fsf.org/wiki/VLC_media_player}
Provee interfaz gráfica y en línea de comandos. 
\item Mplayer. \url{http://www.mplayerhq.hu/}. Principalmente utilizado 
desde el intérprete de comandos, sin embargo existen interfaces gráficas
(GUI) como {\it gmplayer} para utilizarlo. 
\end{itemize}

Las últimas dos aplicaciones, son más versátiles en cuánto a los formatos soportados, 
también reproducen video y son {\it independientes del entorno gráfico}, sin embargo 
su interfaz suele ser menos intuitiva que las primeras cuatro aplicaciones 
mencionadas. 
 
Del mismo modo, existen muchas aplicaciones para controlar el nivel del sonido, micrófono, 
líneas de entrada, etc. Normalmente estarán asociadas al entorno gráfico, es decir suelen ser
un componente de software parte de éste. Sin embargo existen algunas independientes como por ejemplo:
{\tt alsamixer}. Será tarea del administrador identificar dicho software dependiendo 
del entorno utilizado por el usuario. 

Las imágenes a continuación se observan dos ejemplos clásicos de aplicaciones provistas
como parte de entornos gráficos: {\it gnome-sound-recorder} que permite la 
digitalización del sonido recibido desde un micrófono o dispositivo de 
entrada conectado a la tarjeta de sonido; esta aplicación es desarrollada como 
parte del entorno gráfico GNOME; la segunda imagen corresponde a la aplicación 
{\it xfce-mixer} perteneciente al entorno gráfico XFCE, que permite controlar 
distintos parámetros como el volumen de salida de la tarjeta de sonido. 


\begin{figure}[h]
\centering
\includegraphics[width=0.8\textwidth]{gnomerec.png}
\renewcommand{\figurename}{Fig.}
\caption{Gnome Sound Recorder}
\label{contexto:figura}
\end{figure}


\begin{figure}[h]
\centering
\includegraphics[width=0.8\textwidth]{xfcemix.png}
\renewcommand{\figurename}{Fig.}
\caption{XFCE Mixer}
\label{contexto:figura}
\end{figure}

En cuanto a la dependencia del entorno gráfico, es bueno tener en cuenta que 
la misma se establece a nivel de las bibliotecas compartidas, utilizadas
por la aplicación. Entonces, por ejemplo, si estamos utilizando el entorno gráfico 
GNOME y queremos instalar y ejecutar una aplicación que está diseñada para 
ejecutarse en, digamos, KDE, es muy probable que tengamos que instalar 
bibliotecas del entorno KDE y las mismas serán además ejecutadas adicionalmente 
a las utilizadas por GNOME. En este sentido el administrador de sistemas deberá 
inclinarse por sugerir aplicaciones que, o bien sean independientes, o bien 
utilicen bibliotecas ya instaladas, con el fin de minimizar el consumo de 
recursos. 

Si bien la reproducción y control de audio suelen ser los 
requisitos más comunes de los usuarios, podrían surgir requisitos 
más complejos. Mencionamos algunas aplicaciones y su funcionalidad, 
para dar algunos ejemplos: 

\begin{itemize}
\item Audacity \url{https://directory.fsf.org/wiki/Audacity}.  
Permite la grabación y edición de audio. Por ejemplo (contando con el 
hardware adecuado), podríamos utilizarlo para digitalizar audio proveniente
de casetes, o vinilos. Permite importar archivos en distintos formatos (WAV, 
AIFF, AU, FLAC, OGG Vorbis, MP3, etc) y combinarlos, editarlos, convertir a otros 
formatos, etc. También permite varios tipos de edición: eliminar porciones del 
audio, agregar efectos, eliminar ruido, voces, ajustar volumen. También permite 
el análisis de frecuencias mediante espectrogramas. Por mencionar algunas 
de las posibilidades de este software.  
\item LMMS \url{https://directory.fsf.org/wiki/Linux\Multimedia\studio}.
LMMS es un software libre multiplataforma que permite producir música con 
la computadora. Esto cubre la creación de melodías y ritmos, sintetizar 
y mezclar sonidos y organizar muestras, entre otras características. 
\item Qsynth \url{http://qsynth.sourceforge.net/qsynth-index.html}
Es una interfaz gráfica para la aplicación {\it fluidsynth}, que 
implementa un sintetizador\footnote{Software que emula el funcionamiento 
y sonido que producían los antiguos sintetizadores, además de producir 
más modernos y mejores sonidos.} de tiempo real. 
\item Ardour \url{https://directory.fsf.org/wiki/Ardour}. 
Es un grabador de audio multipista / multicanal profesional y DAW. 
{\it Una estación de trabajo de audio digital (EAD) o DAW por sus siglas 
en inglés (Digital Audio Workstation) es un sistema electrónico dedicado
a la grabación y edición de audio digital por medio de un software de 
edición de audio; y del hardware compuesto por una computadora y una 
interfaz de audio digital, encargada de realizar la conversión 
analógica-digital y digital-analógico dentro de la estación}\cite{wikidawes}.
\item libav (avconv). \url{https://libav.org/}. Proporciona herramientas 
multiplataforma y bibliotecas para convertir, manipular y transmitir 
una amplia gama de formatos multimedia y protocolos. En particular 
{\it avconv} es una de estas herramientas, de línea de comandos (CLI), 
que permite distinto tipo de conversiones de audio y video, incluso 
desde una entrada en vivo. También puede convertir entre frecuencias de
muestreo arbitrarias y cambiar el tamaño de vídeo sobre la marcha 
con un filtro polifásico de alta calidad. 
\end{itemize}

Todas las aplicaciones anteriores están orientadas a la edición, conversión 
y producción de audio. Pensando en la manipulación de elementos individuales
que posteriormente formarán (o no) parte de un contenido multimedia más 
complejo. Sin embargo, también existen cientos de posibles requisitos adicionales.
Por ejemplo, los usuarios podrían requerir un sistema de chat de voz como 
el provisto por la aplicación Mumble \url{https://directory.fsf.org/wiki/Mumble}. 
Este tipo de requisitos requieren, además de las cuestiones relativas a la calidad
y el formato, el análisis de los requisitos de transmisión de audio: ancho de banda 
requerido, tipo de red, compresión, etc. 


\subsection*{Video y animación}

En el caso del sonido, hablábamos de una onda sonora
que produce variaciones de presión percibidas por el oído humano, en el caso 
de la imagen en movimiento apelaremos a las habilidades humanas de 
percibir color y luminosidad.  

Con la imagen en movimiento y su captura, el video, sucede algo similar que con 
el sonido, en donde existirán capturas analógicas y digitales, y procesos 
de conversión. El {\it video analógico} es representado como una señal continua, que varía con 
el tiempo, mientras que, el {\it video digital} es representado como una secuencia 
de imágenes digitales. Los formatos analógicos más populares son NTSC, PAL y SECAM.  

El ojo humano retiene la imagen por una fracción de segundo luego de 
que ve la imagen. Esto se conoce como {\it persistencia de la visión}, y es 
una propiedad esencial para todas las tecnologías relacionadas a video y 
animación. La idea será entonces, presentar imágenes individuales estáticas
a una velocidad suficiente para que el ojo humano las integre produciendo 
el movimiento. 

Estás imágenes individuales o {\it fotogramas (frames en inglés)}, y la 
cantidad de ellas mostradas por segundo definen uno de los parámetros a 
tener en cuenta cuando analizamos video: {\bf fps (fotogramas por segundo)}. 
Para darnos una idea del rango de valores que podemos encontrar para 
este parámetro, las viejas cámaras mecánicas cargaban de seis a ocho 
fotogramas por segundo (fps), mientras que se pueden obtener 120 fps o más para 
las nuevas cámaras profesionales. Los estándares PAL (Europa, Asia, Australia, 
etc.) y SECAM (Francia, Rusia, partes de África, etc.) especifican 25 fps, 
mientras que NTSC (EE. UU., Canadá, Japón, etc.) especifica 29,97 fps. El cine 
es más lento con una velocidad de 24fps. Para lograr la ilusión de una 
imagen en movimiento, la velocidad mínima de carga de las imágenes es de unas 
quince imágenes por segundo, sin embargo el ojo humano puede distinguir 
movimiento mucho más fluida por encima de los 48 fotogramas por segundo.

Otros parámetros importantes a tener en cuenta serán la {\bf resolución del video} y
la {\bf relación de aspecto}. En el caso del video digital, la resolución 
se mide en {\it píxeles}, mientras que en el video analógico se mide en 
líneas de barrido horizontal y vertical. 

La relación de aspecto se expresa por la anchura de la pantalla en relación a la altura. 
Por ejemplo una resolución digital de 1280x720 píxeles (ancho x alto) corresponde 
a una relación de aspecto de 1280/720 = 16 / 9. 
El formato estándar hasta el momento en que se comenzó con la estandarización de 
la {\it televisión de alta resolución (HD High Definition)} tenía una relación de 
aspecto de 4/3. Luego, el adoptado es de 16/9. Estas relaciones de aspecto se 
pueden lograr de diferentes formas como se muestra en el gráfico a continuación
para algunas resoluciones estándar conocidas:

\begin{figure}[h]
\centering
\includegraphics[width=0.8\textwidth]{resol.png}
\renewcommand{\figurename}{Fig.}
\caption{Relación de aspecto y resolución}
\label{contexto:figura}
\end{figure}

Mencionaremos también, como parámetro a tener en cuenta en lo que a video respecta
y, como una extensión de lo mencionado para imágenes estáticas, la {\bf cantidad de 
bits por píxel bpp}. Esto determinará la cantidad de colores que un píxel 
puede representar. 

Teniendo en cuenta lo anterior, el espacio de almacenamiento del video digital 
será proporcional a la resolución, los fotogramas por segundo (fps) y la 
profundidad de color de cada fotograma (bpp). Esto es opuesto a lo que sucede
con la calidad del video obtenida al reducir alguno de los parámetros mencionados.

Yendo a los contenidos multimedia, para que un video sea utilizable dentro de 
una composición multimedia, éste debe encontrarse en formato digital. En este 
sentido el video puede ser más complejo que el sonido a la hora de digitalizar
formatos analógicos, debido a la multiplicidad de fuentes posibles. Por ejemplo, 
debido a la gran cantidad de video en cinta disponible (ej VHS), muchos videos 
comienza en formato analógico. El pasaje a formato digital requerirá del 
equipamiento necesario para reproducir las cintas originales, cables y conectores, 
hardware de captura de video en la computadora donde se pretende producir la 
digitalización. Existen otras posibilidades, por ejemplo podríamos querer 
capturar la imagen analógica proveniente de la señal de cable, etc. Hoy en día 
existen existen cámaras de video (filmadoras) digitales, que almacenan directamente
la imagen digitalizada reduciendo los requisitos de hardware y software, en este
sentido. La siguiente imagen corresponde a una tarjeta de extensión PCIe que posee
diversos conectores de entrada y salida de video: 


\begin{figure}[h]
\centering
\includegraphics{PLACA.jpg}
\renewcommand{\figurename}{Fig.}
\caption{DeckLink Studio 4K Capture \& Playback Card from Blackmagic Design}
\label{contexto:figura}
\end{figure}

Dentro de los requisitos de hardware, los conectores clásicos para video más 
populares son:


\begin{itemize}
\item RCA - conector Cinch (Radio Corporation of America). Es antiguo y el menos 
común en tarjetas de video. Sin embargo, aún se utiliza con mayor frecuencia 
en equipos de audio y video (por ejemplo para conectar un televisor a un 
reproductor de VCR o DVD).Los conectores RCA se utilizan con mayor frecuencia 
para transmitir señales analógicas. Un determinado cable RCA sólo puede llevar 
una señal a la vez. Por lo tanto, se deben usar múltiples conectores RCA para 
ofrecer todos los componentes de una señal. Por ejemplo, el video de una 
película podría ser llevado a través de un cable RCA en un conector, 
mientras que el sonido de la misma película tendría que ser transportado en 
un cable separado de RCA. 
\item S-Video - Separated-Video. Transmite una señal analógica de video. 
Se usa con frecuencia en televisores, reproductores de DVD, grabadores de 
video, y videoconsolas modernas. Muchas tarjetas gráficas y tarjetas 
sintonizadoras de TV también tienen, respectivamente, salida y entrada 
de S-Video. También es muy común encontrar el conector S-Video en 
computadoras portátiles.
\item VGA - Video Graphics Array. Ha sido el conector más utilizado en los 
últimos tiempos, la gran mayoría de las tarjetas de video lo proveen. Sin embargo
fue diseñado para señales analógicas y display CRT (tubo de rayos catódicos). 
Está siendo gradualmente reemplazado por HDMI. 
\item Mini-VGA - Mini-VGA. El conector mini-VGA se utiliza en equipos 
portátiles y otros sistemas, en lugar del VGA estándar. Además de su 
tamaño compacto, el puerto mini-VGA también permite una señal de video compuesta 
y S-Video, además de la señal de VGA.  Hoy en día, el conector mini-DVI ha 
sustituido el mini-VGA.
\item DVI - Digital Visual Interface. Diseñada para video digital a 
diferencia de VGA que fue diseñada para visualización en CRT (tubo de 
rayos catódicos, viejos monitores). Soporta compatibilidad con VGA, sin 
embargo HDMI se plantea como su sucesor.  
\item HDMI - High-Definition Multimedia Interface. Es más reciente que 
DVI, no tiene compatibilidad con VGA. Se propone como sucesor de 
DVI, presenta características adicionales sobre éste. 
\end{itemize}

\begin{figure}[h]
\centering
\includegraphics[width=0.8\textwidth]{conectores.png}
\renewcommand{\figurename}{Fig.}
\caption{Derecha a izquierda: RCA, SVideo, DVI, HDMI, VGA y Mini-VGA}
\label{contexto:figura}
\end{figure}

Existen multiplicidad de requisitos posibles y aplicaciones. En este sentido 
mencionaremos algunas aplicaciones que permiten requisitos básicos relacionados
a video como ser: captura, reproducción y edición. Adicionalmente, mencionamos 
algunas aplicaciones que permiten componer animaciones 2D y 3D. 

{\bf Reproducción:}
\begin{itemize}
\item Totem.   \url{https://directory.fsf.org/wiki/Totem}. GNOME. 
\item VLC y Mplayer.  Ambos mencionados en la sección audio, también aplican 
a la reproducción de video en varios formatos.  
\end{itemize}

{\bf Edición:}
\begin{itemize}
\item PiTiVi. \url{https://directory.fsf.org/wiki/PiTiVi}.
Permite edición de audio y video. Basado en la biblioteca para
streaming de medios, Gstreamer (\url{https://directory.fsf.org/wiki/Gstreamer})
\item OpenShot. \url{https://directory.fsf.org/wiki/OpenShot}. Edición 
de audio y video. Permite agregar. sub-titles, transiciones, y efectos, para 
luego exportar el video a DVD, YouTube, Vimeo, Xbox 360, y otros formatos comunes.
\item Avidemux 
\item Cinelerra. \url{https://directory.fsf.org/wiki/Cinelerra}. Edición de audio
y video, permite captura y reproducción. 
\end{itemize}


{\bf Animación:}
\begin{itemize}
\item Blender. \url{https://directory.fsf.org/wiki/Blender}. Es una suite de
animación  3D integrada, para modelado, animación, rendering, post-producción, creación 
interactiva y reproducción (juegos). 
\item Stopmotion. \url{https://directory.fsf.org/wiki/Stopmotion}. Permite
crear animaciones stop-motion\footnote{Técnica de animación que consiste 
en aparentar el movimiento de objetos estáticos por medio de una serie de 
imágenes fijas sucesivas.}, agregar audio, y luego exportarlas a 
diferentes formatos (ej. MPEG, avi). 
\item Synfig Studio. \url{https://directory.fsf.org/wiki/Synfig_Studio}. 
Animación 2D, basado en imagen vectorial y matricial. 
\item Tupí. \url{http://www.maefloresta.com/portal/es/acerca}. Animación
2D. 
\end{itemize}

\textcolor[rgb]{1,0,0}{NOTA: las secciones a continuación se encuentran en 
construcción}

\begin{figure}[h]
\centering
\includegraphics{UN_CONSTRUCTION_2ss.png}
\renewcommand{\figurename}{Fig.}
\label{contexto:figura}
\end{figure}

\subsection*{Autoría multimedia (Authoring)}

Hemos desarrollado brevemente aplicaciones multimedia relacionadas a la 
creación y manipulación de componentes individuales que pueden luego 
dar lugar a una aplicación multimedia. Sin embargo, para producir 
una aplicación multimedia efectiva, estos componentes deben ser 
integrados, y cierto grado de interactividad añadido, para permitir
al usuario control sobre la experiencia. 

Esta sección esta enfocada en las herramientas que permiten componer una 
presentación multimedia, incluyendo 
control interactivo por parte del usuario. Son herramientas que vinculan 
elementos individuales de imagen, gráfico, audio, video, animación y texto. 
A este proceso creativo de composición se lo denomina {\it autoría multimedia} 
(del inglés``multimedia authoring''). 

Los programas de autoría multimedia ({\it authoring tools}) ayudan al desarrollador
a ensamblar la composición multimedia y agregar la interacción necesaria. 

El grado de interacción necesaria dependerá de la aplicación de la 
composición. El espectro va desde casi ninguna interacción, como puede 
ser una presentación de LibreOffice Impress (slideshow), hasta la inmersión 
total en entornos de realidad virtual o aumentada. 

En un slideshow la interactividad consiste generalmente, en el avance 
paso a paso entre slides. El siguiente nivel de interactividad, será ser
capaz de controlar la secuencia y decidir dónde avanzar a partir de 
un conjunto de posibilidades (por ejemplo lo que sucede con una wiki). 
Lo siguiente será el control del medio: detener/reproducir video, 
buscar un texto, desplazamientos y zoom sobre la vista, etc.  
El nivel de interacción puede ser mayor si se agrega control de objetos, 
como se puede dar en un video juego, realidad aumentada, etc. 


Las características clásicas de los programas de autoría que pueden 
ser requeridas por los usuarios son: 

\begin{itemize}
\item Capacidad para crear varios tipos de componentes multimedia individuales.
\item Capacidad para importar componentes multimedia.
\item Capacidad para integrar los componentes: ubicar, enlazar, secuenciar. 
\item Capacidad para incorporar interactividad. 
\item Capacidad para producir producir una aplicación multimedia empaquetada, 
lista para su uso. 
\end{itemize}

La selección de un programa de autoría dependerá de la audiencia pretendida
para la composición multimedia, el medio en el cual se va a presentar (web, 
CD, presentación, etc). Y no menos importante la curva de aprendizaje 
requerida por la aplicación de autoría. En este sentido, la creación de 
presentaciones usando LibreOffice Impress puede ser bastante intuitiva, 
mientras que el uso de un software comercial como Adobe Director requerirá
de más entrenamiento, previo a desarrollar una aplicación multimedia.  

%Existen cuatro tipos de autoría multimedia: 
%\begin{itemize}
%\item Autoría multimedia basada en metáforas: se trata de un conjunto de 
%interfases visuales, acciones y procedimientos que explotan 
%conocimientos específicos que los usuarios ya poseen acerca de otros dominios. 
%El propósito de la interfaz, metáfora, es darle al usuario conocimiento 
%instantáneo acerca de cómo usar la interfaz para crear aplicaciones multimedia. 
%\item Producción multimedia
%\item Presentación multimedia 
%\item Autoría automática
%\end{itemize} 
%
%{\bf Metáfora de autoría multimedia}
%
%La metáfora de autoría, o paradigma de autoría, es la metodología por 
%la cual el sistema de autoría lleva a cabo su tarea. Hay varias 
%metáforas: 
%
%\begin{itemize}
%\item Metáfora lenguaje de scripting: utiliza un lenguaje de 
%programación especial para proveer interactividad (botones, mouse, etc.), 
%saltos, condicionales, bucles, funciones, macros, etc.  
%\item Metáfora de slideshow: se trata de presentaciones lineales, si bien existen 
%herramientas que permiten saltos entre slides. Ejemplos pueden ser Inkscape con su 
%extensión Sozi y LibreOffice Impress. 
%\item Metáfora jerárquica: los elementos controlables por los usuarios se 
%organizan en una estructura de árbol, con frecuencia utilizado en las aplicaciones
%dirigidas por menú. 
%\item Metáfora basada en íconos y control de flujo: se dispone de íconos gráficos 
%en una caja de herramientas, y el proceso de autoría consiste en ir creando un 
%diagrama de flujo utilizando dichos íconos. Authorware
%\item Metáfora de cuadros (frames): similar al anterior, pero los vínculos entre 
%íconos son más conceptuales, en lugar de representar el flujo real del programa.
%Otra de los formas en que las herramientas de autoría funcionan es a través del
%modelado de una pila de cuadros (o páginas de un libro) . En este caso, cada cuadro
%representa una única vista en la pantalla. Cuando el usuario se mueve a otro cuadro, 
%la pantalla cambia. Cada cuadro puede contener varios componentes, tales como texto, 
%gráficos y video.
%\item Metáfora de tarjetas y scripting: 
%\item Metáfora Cast/Score/Scripting. Adobe Director es el principal ejemplo
%comercial de este tipo. 
%\end{itemize}

Comparativa para autoría de e-learning:
\url{http://oss-watch.ac.uk/resources/ossoptionseducation}


\begin{itemize}
\item{Wiki}
\item{Openmeeting}
\item{PureData} \url{http://puredata.info/}
\item{Ofimática:presentaciones,textos,etc.}
\item{Inkscape+sozi}
\end{itemize}

\begin{thebibliography}{1}

\bibitem{raemm}
Real Academia Española, \emph{Multimedia}.\hskip 1em plus
0.5em minus 0.4em\url{http://lema.rae.es/drae/?val=multimedia}

\bibitem{ramyer}
Ramesh Yerraballi, \emph{Multimedia Systems, Concepts Standards and 
Practice}.\hskip 1em plus  0.5em minus 0.4em
\url{http://users.ece.utexas.edu/~ryerraballi/MSB/Contents.html}

\bibitem{frankuo}
Franklin F. Kuo, J. Joaquin Garcia Luna-Aceves y Wolfgang Effelsberg. 
\emph{Multimedia Communications: Protocols and Applications}.\hskip 1em plus
  0.5em minus 0.4em\relax Publisher: Prentice Hall; ISBN-13: 978-0138569235  

\bibitem{wikipmmes}
Wikipedia en español. \emph{Multimedia}.\hskip 1em plus
  0.5em minus 0.4em\url{http://es.wikipedia.org/wiki/Multimedia}

\bibitem{wikipmmen}
Wikipedia en inglés. \emph{Multimedia}.\hskip 1em plus
  0.5em minus 0.4em\url{http://es.wikipedia.org/wiki/Multimedia}

\bibitem{wikiives}
Wikipedia en español. \emph{Gráfico Vectorial}.\hskip 1em plus
  0.5em minus 0.4em\url{http://es.wikipedia.org/wiki/Gráfico\_vectorial}

\bibitem{wikiimes}
Wikipedia en español. \emph{Imagen de mapa de bits}.\hskip 1em plus
  0.5em minus 0.4em\url{http://es.wikipedia.org/wiki/Imagen\_de\_mapa\_de\_bits}

\bibitem{wikidawes}
Wikipedia en español. \emph{Estación de trabajo de audio digital}.\hskip 1em plus
  0.5em minus 0.4em\url{http://es.wikipedia.org/wiki/Estaci\%C3\%B3n\_de\_trabajo\_de\_audio\_digital}


\end{thebibliography}{1}

\section*{Licencia}
Copyright (C) 2014 Lechner Miriam.

Se concede autorización para copiar, distribuir y/o modificar este documento
bajo los términos de la Licencia Creative Commons Atribución-CompartirDerivadasIgual 3.0 Unported. 

http://creativecommons.org/licenses/by-sa/3.0/

\end{document}

\end{thebibliography}{1}

\section*{Licencia}
Copyright (C) 2014 Lechner Miriam.

Se concede autorización para copiar, distribuir y/o modificar este documento
bajo los términos de la Licencia Creative Commons Atribución-CompartirDerivadasIgual 3.0 Unported. 

http://creativecommons.org/licenses/by-sa/3.0/

\end{document}
