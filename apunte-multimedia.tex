%%% LaTeX Template: Article/Thesis/etc. with colored headings and special fonts
%%%
%%% Source: http://www.howtotex.com/

\documentclass[12pt]{article}


\usepackage{apuntes-estilo}
\usepackage{fancyhdr,lastpage}
\usepackage{color,colortbl}
\usepackage{verbatim}

\def\maketitle{

% Titulo 
 \makeatletter
 {\color{bl} \centering \huge \sc \textbf{
  Aplicaciones Multimedia\\ 
\large \vspace*{-8pt} \color{black}Introducción a aplicaciones multimedia. 
 \vspace*{8pt} }\par}
 \makeatother

% Autor
\makeatletter
 {\centering \small 
 	Departamento de Ingeniería de Computadoras \\
 	Facultad de Informática - Universidad Nacional del Comahue \\
 	\vspace{20pt} }
 \makeatother

}

% Custom headers and footers
\fancyhf{} % clear all header and footer fields
\fancypagestyle{plain}{\fancyhf{}}
  	\pagestyle{fancy}
 	\lhead{\footnotesize Aplicaciones multimedia - Departamento de Ingeniería de Computadoras}
 	\rhead{\footnotesize \thepage\ }	% ''Page 1 of 2''

\def\ti#1#2{\texttt{#1} & #2 \\ }



\begin{document}

\thispagestyle{empty}
\maketitle
\setlength{\parindent}{0pt}

\section*{Introducción}

Para comprender el alcance de la palabra multimedia analizaremos distintas 
definiciones. Según la Real Academia Española, multimedia se refiere a un 
adjetivo que indica: ``que utiliza conjunta y simultáneamente diversos 
medios, como imágenes, sonidos y texto, en la transmisión de una 
información.''\cite{raemm}

\begin{center}
\includegraphics{multimedia.png}
\end{center}

Desde un punto de vista etimológico podemos decir que multimedia se 
refiere a: multi y media. Tal que multi se refiere a  muchos, y media a: 
substancia intermedia a través de la cual algo es transmitido o 
transportado. Un medio de comunicación masivo, tal como un periódico o 
la televisión.\cite{ramyer}

Estas definiciones nos das un primer acercamiento al alcance de la palabra
multimedia. Sin embargo, si nos preguntamos qué aspectos de este alcance 
serían importantes cubrir desde el punto de vista de {\it aplicaciones 
multimediales}, o desde la realidad de un administrador de sistemas y un 
entorno de trabajo convencional, como puede ser una oficina, todavía queda
mucho por aclarar. 

En este sentido la definición de Franklin Kuo, establece dos aspectos del
concepto de multimedia en los cuales haremos foco:

\begin{itemize}
\item Multimedia se refiere a la representación de medios mixtos de información
-texto, datos, imágenes, audio y video- como señales digitales. 
\item Comunicaciones multimedia, se refiere a la tecnología requerida para 
manipular, transmitir y controlar estas señales audiovisuales a través de un 
canal de comunicación.\cite{frankuo}
\end{itemize}


Teniendo los dos aspectos mencionados por Kuo, si nos concentramos en contenidos 
multimedia dentro computadora aislada, es decir,  sin conexión a red (ya sea LAN o 
Internet), entonces nos enfocaremos por un lado en la representación y almacenamiento 
de datos multimediales, utilizando para ello los conocimientos previos 
adquiridos sobre formato interno de archivos regulares. Por otro lado, en 
las {\it aplicaciones multimedia} que nos permiten {\it generar} y manipular 
dichos contenidos localmente. 

Sin embargo, dado la expansión de Internet, y el frecuente desarrollo 
de redes de área local (LAN), debemos considerar el segundo aspecto mencionado
por Kuo referente a las {\it Comunicaciones multimediales}. La transmisión 
de datos multimediales sin dudas representa un desafío para los administradores
de sistemas en cuánto al análisis de recursos utilizados en dicha transmisión.
Una transmisión de video en vivo que no llega a tiempo a sus destinos, será 
una comunicación multimedial no exitosa, por lo que el análisis de los 
recursos necesarios se vuelve una tarea compleja e imprescindible en los 
sistemas modernos. 

Por último y nos menos importante, abordaremos brevemente los aspectos legales 
sobre la generación y transmisión de contenidos multimedia. En muchos casos,
hemos naturalizado el uso de imágenes, sonidos, textos y videos obtenidos 
desde Internet, sin prácticamente cuestionarnos acerca de la validez 
legal del uso que hacemos. Quizá es más evidente cuando se trata de 
películas o audio, ya que existen multiplicidad de campañas que, mal o 
bien, intentan introducir las consideraciones legales acerca de la descarga y 
reproducción de estos contenidos. Sin embargo, cuántos de nosotros nos 
preguntamos, por ejemplo al desarrollar el contenido de una wiki, si la 
imagen que acabamos de incorporar puede ocasionar un problema legal. 

En resumen, abordaremos multimedia siguiendo tres pilares:
\begin{itemize}
\item Aspectos estáticos: creación, reproducción (aplicaciones y formatos) y 
consumo de recursos, como ser almacenamiento y requisitos de procesamiento. 
\item Aspectos dinámicos: transmisión dentro de redes de área local (LAN) e 
Internet. 
\item Aspectos legales.  
\end{itemize}

\section*{Usos}

Los diferentes formatos de multimedia analógica o digital tienen la 
intención de mejorar la experiencia de los usuarios, por ejemplo para 
que la comunicación de la información sea más fácil y rápida. O, en 
el entretenimiento y el arte, para trascender la experiencia común.\cite{wikipmmes} 

Los usos de la multimedia digital son potencialmente ilimitados. Algunos 
ejemplos de uso que podríamos mencionar son:  

\begin{itemize}
\item {\bf Uso comercial}: publicidad, artistas que distribuyen sus obras como contenidos 
multimediales digitales, venta de productos a través de catálogos en la web, capacitaciones online. 
\item {\bf Entretenimiento y bellas artes}: nuevos desarrollos artísticos y nuevas formas 
de presentación del arte existente, como las galerías de arte online (Google Art Project, 
Museo del Prado, etc) han cobrado vida en los últimos años.  
\item {\bf Educación}: PEDCO podría ser un ejemplo, enciclopedias interactivas, entrenamientos
interactivos, etc. 
\item {\bf Periodismo}: La explosión de los periódicos digitales, la incorporación de dispositivos 
multimediales en los noticieros televisivos, son algunos ejemplos. 
\item {\bf Ingenierías, arquitectura, ciencias en general, medicina}: todas las ciencias se ven 
beneficiadas por el uso de multimedia. La simulación y el modelado son un recurso 
valioso en todas estas áreas del conocimiento. 
\end{itemize}


\section*{Categorizaciones}

Desarrollaremos brevemente dos categorizaciones, ambas relacionadas a la 
percepción del usuario con respecto al contenido multimedia. En un caso 
nos referiremos a la interactividad del usuario; mientras que en el otro 
haremos referencia al componente temporal que puede o no afectar a la
reproducción de los contenidos. 

A la hora de evaluar contenidos multimedia, tendremos en cuenta entonces
por un lado los mecanismos de interacción (si los hubiere), y por otro 
cómo afecta el tiempo a la reproducción de elementos individuales. 

\subsection*{Interactividad}
En cuanto a la interacción que tiene el usuario con un contenido 
multimedial (como un todo, es decir un conjunto de elementos individuales
como texto, imagen, etc), podemos dividir dichos contenidos en: 
\begin{itemize}
\item Lineal: la reproducción de este tipo de contenidos progresa 
sin intervención del observador. Por ejemplo, un presentación cinematográfica.  
\item No lineal: en este caso, la interactividad es utilizada para 
controlar el progreso de la reproducción del contenido multimedial. Por 
ejemplo un video juego, una enciclopedia virtual en donde el observador
avanza a través de los contenidos que prefiere (ejemplo de hypermedia, 
wikipedia es una implementación posible), etc. 
\end{itemize}

\subsection*{Tipos de medios}
Por otra parte, teniendo en cuenta los elementos individuales que componen un 
contenido multimedia, podemos podemos distinguir dos grandes tipos de medios: 

\begin{itemize}
\item{Estáticos, medio discreto independiente del tiempo}: 
Dentro de esta clase encontramos a las imágenes y gráficos fijos, y texto. La información 
en este medio consiste exclusivamente de una secuencia de elementos individuales 
sin un componente de tiempo.
\item{Dinámicos, medio continuo dependiente del tiempo}: 
Aquí encontramos al sonido, animaciones y video. La información no sólo es 
expresada por su valor individual, sino que también por el momento 
de su ocurrencia.
\end{itemize}

Claramente los medios continuos representarán un mayor desafío para el 
administrador a la hora de la transmisión a través de una red. En este 
sentido también podemos subdividir la clasificación en {\it transmisión
en vivo o grabada}. En ambos casos es posible la interactividad, pero 
la transmisión en vivo impone aún más restricciones y desafíos para 
considerarse exitosa. 

Una aclaración importante acerca de esta calcificación es que la noción 
de {\it dependencia del tiempo} no tiene relación con la representación
del dato, en el sentido de que ambos medios, discretos y continuos, en su 
representación física puede no tener información acerca del componente 
temporal. El aspecto temporal en este caso, es más bien la percepción del 
que escucha u observa.\cite{ramyer}  

\section*{Aplicaciones multimedia}

Podríamos comenzar esta sección realizando una lista de aplicaciones 
libres conocidas para multimedia y hacer una descripción breve de cada
una de ellas. Sin embargo además de caer en la obsolesencia mientras 
leemos este apunte, sería imposible ser exaustivos. Utilizaremos sólo
un breve listado a modo de ejemplo y que al momento de escribir este 
apunte gozan de cierta popularidad.  

El objetivo será entonces, caracterizar las aplicaciones pensando
siempre en nuestro rol de administradores de sistema y, en particular 
en el asesoramiento que usualmente nos es demandado por los usuarios. No
será nuestro objetivo ahondar en el diseño multimedial, sino obtener 
los conocimientos básicos para poder entender la problemática, sugerir 
posibles soluciones o simplemente ser el punto de partida para un análisis 
más profundo cuando el entorno a administrar así lo demande. 

En cada caso trataremos de identificar: 
\begin{itemize}
\item Funcionalidad: ¿Sirve para producir elementos individuales (discretos o contínuos) o 
bien es una herramienta que permite la vinculación y presentación de elementos individuales? 
\item Consumo de recursos. Necesidades de la aplicación con respecto al 
hardware y al sistema operativo. Por ejemplo ciertas aplicaciones para generar 
elementos de animación requieren hardware gráfico específico, otras pueden demandar 
una conexión de red para transmitir el contenido, etc. 
\end{itemize}

\subsection*{Creación y manipulación de gráficos e imágenes estáticas}

Una problemática frecuente es la necesidad de generar y manipular gráficos e imágenes 
bidimensionales. Por citar algunos de los infinitos riquisitos posibles, un usuario 
podría necesitar: 
\begin{itemize}
	\item Editar fotos. El usuario desea retocar una o cientos de imágenes capturadas
	      con alguna cámara digital, o escaneadas. En este caso no sólo debemos pensar en el problema de 
	      encontrar software que le permite realizar la edición, sino que también debemos
	      pensar en el mecanismo que le permita importar las fotos como imágenes digitales
	      dentro de su computadora para su posterior edición.    
	\item Creación de imágenes matriciales e imágenes matriciales multicapa (como las generadas por 
	Adobe Photoshop o Gimp).  
	\item Cambiar el formato de una imágen. Por ejemplo pasar uno o cientos de archivos de 
	un formato PNG a JPG. Además podría requerir asesoramiento de qué formato sería el 
	indicado si tuviese que visualizarlo en otros sistemas operativos. 
	\item Crear un diagrama vectorial. Por ejemplo el usuario crear un mapa conceptual, o
	un organigrama jerárquico que represente a la organización en la que trabaja.    
	\item Crear una gráfico vectorial (como caso general de lo anterior). Por ejemplo el 
	usuario podría querer crear un logo para la empresa que pueda visualizarce luego en 
	una página web.  
	\item Tomar una captura de pantalla y editarla para luego incorporarla en una presentación. 
\end{itemize}

Deberemos tener presente al menos que existen: {\bf{\it imágenes matriciales e imágenes vectoriales}}. 
En este sentido:  

``Una imagen vectorial es una imagen digital formada por objetos geométricos independientes (segmentos, 
polígonos, arcos, etc.), cada uno de ellos definido por distintos atributos matemáticos de forma, de 
posición, de color, etc. Por ejemplo un círculo de color rojo quedaría definido por la posición 
de su centro, su radio, el grosor de línea y su color.

Este formato de imagen es completamente distinto al formato de las imágenes de mapa de bits, también 
llamados imágenes matriciales, que están formados por píxeles. El interés principal de los gráficos 
vectoriales es poder ampliar el tamaño de una imagen a voluntad sin sufrir la pérdida 
de calidad que sufren los mapas de bits. De la misma forma, permiten mover, estirar y 
retorcer imágenes de manera relativamente sencilla.'' \cite{wikiives}

``A las imágenes en mapa de bits se las suele definir por su altura y anchura (en píxeles) y por 
su profundidad de color (en bits por píxel), que determina el número de colores distintos que se 
pueden almacenar en cada punto individual, y por lo tanto, en gran medida, la calidad del color de la imagen.''
\cite{wikiimes}

A esto podemos adicionar la existencia de formatos de imagen mas complejos que involucran las antes mencionadas
como base, como ser XCF de Gimp, o PSD de Adobe Photoshop. 

Analizando un poco el consumo de recursos, en líneas generales una imágen matricial ocupa mayor espacio de
almacenamiento que su equivalente vectorial. En el primer caso, el tamaño es proporcional al 
tamaño de la matriz de píxeles, mientras que en el segundo caso estamos hablando de representaciones
matemáticas que no varían con el tamaño visualizado.  

Si bien el procesamiento en masa, por ejemplo el escalado de cientos de imágenes, podría hacerse 
sin la necesidad de un entorno gráfico (por ejemplo utilizando el programa convert). 
En general, la mayoría de las aplicaciones gráficas demandarán al menos: hardware gráfico (vulgarmente una placa de video); 
monitor para visualización; cierta capacidad de procesamiento (CPU) y memoria principal (RAM), 
ya que las aplicaciones gráficas aún cuando nos referimos a imagenes bidimensionales, suelen consumir 
bastantes recursos. Mecanismos de interacción. Elementos de captura de imágenes: cámaras digitales, 
escanners, etc.  


Gimp
https://directory.fsf.org/wiki/GIMP

Inkscape

 

Mencionaremos tres aplicaciones: 
Imagen y gráficos fijos: Gimp, Día, Inkscape (sozi)
Animación: Tupí, Stopmotion (http://linuxstopmotion.org), Blender, 
Entangle (http://entangle-photo.org/).
Sonido:  

\textbf{}
\subsection*{Aplicaciones para presentar/reproducir contenidos multimediales}
Wiki
Audio/Video/Animación: Vlc, Mplayer. 
Openmeeting
Ofimática: textos, presentaciones, etc. 

\subsection*{Análisis de recursos}
CPU/RAM/IO (Almacenamiento)

\section*{Transmisión}


\section*{Aspectos Legales}

\begin{thebibliography}{1}

\bibitem{raemm}
Real Academia Española, \emph{Multimedia}.\hskip 1em plus
0.5em minus 0.4em\url{http://lema.rae.es/drae/?val=multimedia}

\bibitem{ramyer}
Ramesh Yerraballi, \emph{Multimedia Systems, Concepts Standards and 
Practice}.\hskip 1em plus  0.5em minus 0.4em
\url{http://users.ece.utexas.edu/~ryerraballi/MSB/Contents.html}

\bibitem{frankuo}
Franklin F. Kuo, J. Joaquin Garcia Luna-Aceves y Wolfgang Effelsberg. 
\emph{Multimedia Communications: Protocols and Applications}.\hskip 1em plus
  0.5em minus 0.4em\relax Publisher: Prentice Hall; ISBN-13: 978-0138569235  

\bibitem{wikipmmes}
Wikipedia en español. \emph{Multimedia}.\hskip 1em plus
  0.5em minus 0.4em\url{http://es.wikipedia.org/wiki/Multimedia}

\bibitem{wikipmmen}
Wikipedia en inglés. \emph{Multimedia}.\hskip 1em plus
  0.5em minus 0.4em\url{http://es.wikipedia.org/wiki/Multimedia}

\bibitem{wikiives}
Wikipedia en español. \emph{Gráfico Vectorial}.\hskip 1em plus
  0.5em minus 0.4em\url{http://es.wikipedia.org/wiki/Gráfico\_vectorial}

\bibitem{wikiimes}
Wikipedia en español. \emph{Imagen de mapa de bits}.\hskip 1em plus
  0.5em minus 0.4em\url{http://es.wikipedia.org/wiki/Imagen\_de\_mapa\_de\_bits}

\end{thebibliography}{1}

\section*{Licencia}
Copyright (C) 2014 Lechner Miriam.

Se concede autorización para copiar, distribuir y/o modificar este documento
bajo los términos de la Licencia Creative Commons Atribución-CompartirDerivadasIgual 3.0 Unported. 

http://creativecommons.org/licenses/by-sa/3.0/

\end{document}

\end{thebibliography}{1}

\section*{Licencia}
Copyright (C) 2014 Lechner Miriam.

Se concede autorización para copiar, distribuir y/o modificar este documento
bajo los términos de la Licencia Creative Commons Atribución-CompartirDerivadasIgual 3.0 Unported. 

http://creativecommons.org/licenses/by-sa/3.0/

\end{document}
