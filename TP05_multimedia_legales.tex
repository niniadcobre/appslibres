%%% LaTeX Template: Article/Thesis/etc. with colored headings and special fonts
%%%
%%% Source: http://www.howtotex.com/

\documentclass[12pt]{article}


\usepackage{apuntes-estilo}
\usepackage{fancyhdr,lastpage}



\def\maketitle{

% Titulo 
 \makeatletter
 {\color{bl} \centering \huge \sc \textbf{
Trabajo práctico N 5 \\
\large \vspace*{-8pt} \color{black} Aspectos Legales de las producciones Multimedia 
 \vspace*{8pt} }\par}
 \makeatother


% Autor
 \makeatletter
 {\centering \small 
	Aplicaciones Libres\\
 	Departamento de Ingeniería de Computadoras \\
 	Facultad de Informática - Universidad Nacional del Comahue \\
 	\vspace{20pt} }
 \makeatother

}

% Custom headers and footers
\fancyhf{} % clear all header and footer fields
\fancypagestyle{plain}{\fancyhf{}}
  	\pagestyle{fancy}
 	\lhead{\footnotesize TP N 5 - Aplicaciones Libres}
 	\rhead{\footnotesize \thepage\ }	% "Page 1 of 2"

\def\ti#1#2{\texttt{#1} & #2 \\ }



\begin{document}

\thispagestyle{empty}
\maketitle
\setlength{\parindent}{0pt}


\section*{Introducción}


Las composiciones multimedia, traen aparejado la necesidad de evaluar ciertos
aspectos legales que pueden afectar a la organización que la produce (o 
individuo). Por ejemplo, una presentación como las que podemos realizar con 
LibreOffice Impress, podría traer aparejado un conflicto legal para la 
organización si su contenido digital no esta correctamente verificado. 

En grandes organizaciones, existen departamentos especializados en lo referente 
a aspectos legales, que establecen las políticas a seguir cuando los empleados 
pretenden crear contenidos multimedia en nombre de la misma. De ese modo,
los usuarios deberán respetar tal o cual licencia sobre su producción, podrán o 
no utilizar el logo de la empresa, etc. Un empleado podría preguntarse por 
ejemplo, si puede utilizar un video en su presentación, que acaba de descargar
de la web.  

En organizaciones más pequeñas, muchas veces este departamento especializado no
existe, y las producciones multimedia de los usuarios se crean sin ningún tipo 
de asesoramiento legal. En estos casos, muchas veces recae en el administrador
o gerente de sistemas, alertar a los usuarios sobre las conductas que pueden 
generar un conflicto legal a la hora de crear producciones multimedia. Porque 
después de todo, fueron los administradores los que instalaron el software que
terminó dando origen a la producción :) (aunque lamentable, muchas veces esta 
es la asociación que padecen los administradores). 

Por un lado está la legalidad del software utilizado para la composición 
multimedia. Este aspecto legal ha sido cubierto en otras materias, y atañe
tanto a administradores como a usuarios. En líneas generales, debemos respetar
cada licencia de cada software utilizado para la composición multimedia 
(desde los editores, reproductores, software de autoría etc). 

Pero más complejo aún es lo referente a los derechos de autor sobre la 
composición multimedia en sí misma, y cada elemento individual que la 
compone: texto, imagen, video, etc. 

\section*{Descripción}
La intención de este cuestionario es abrir el debate para alertarnos 
acerca de la complejidad de este problema y vislumbrar ideas acerca de cómo 
prevenir conflictos legales derivados de las producciones multimedia. 
Con este fin, analizaremos el documental producido por TVE (
Televisión Española), llamado Copiad Malditos: 

\url{http://copiadmalditos.blogspot.com.ar/p/videos-el-documental.html}


\begin{itemize}
\item ¿Qué es el derecho de autor?

\item ¿Qué es el copyright?

\item ¿Qué es el copyleft?

\item ¿Cuál es el organismo nacional equivalente a SGAE de España?

\item ¿Qué es creative commons? 

\item ¿Qué implican las licencias creative commons?

\item ¿Hay otras licencias de tipo copyleft para producciones multimedia?

\item ¿Qué cree que sucede con los derechos de auto dentro de las grandes 
corporaciones y empresas privadas? 

\item En el caso de la producción multimedia del documental de TVE, ¿Importa sólo
el derecho de autor de quien hace la composición en su conjunto o se ven 
involucrados los derechos de los productores de cada elemento individual?

\item Piense en sus propias producciones multimedia, por ejemplo todos los 
documentos creados para esta materia en la wiki. ¿Tiene Ud. derecho sobre
cada imagen, texto, video, etc. utilizado en su producción? ¿Pensó que 
podría producir un conflicto a la Universidad por utilizar elementos sobre
los cuales no tiene derechos? Observe que la wiki esta disponible al público
en general a través de Internet, bajo un dominio (fi.uncoma.edu.ar) controlado
por la Universidad.
\end{itemize}

\end{document}
